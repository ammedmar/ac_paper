%%%%%%%%%%%%%%%%%%%%%%
\documentclass{amsart}
% !TEX root = ../ac_paper.tex

% General packages
\usepackage{microtype}
\usepackage{amsmath, amssymb, amsthm}
\usepackage{mathtools}
\usepackage{tikz-cd}
\usepackage{mathbbol} % changes \mathbb{} and adds more support

% Complex configurations
% bibliography
\usepackage[
backend=biber,
style=alphabetic,
backref=true,
url=false,
doi=false,
isbn=false,
eprint=false]{biblatex}

\renewbibmacro{in:}{} % don't display "in:" before the journal name
\AtEveryBibitem{\clearfield{pages}} % don't show page numbers

\DeclareFieldFormat{title}{\myhref{\mkbibemph{#1}}}
\DeclareFieldFormat
[article,inbook,incollection,inproceedings,patent,thesis,unpublished]
{title}{\myhref{\mkbibquote{#1\isdot}}}

\newcommand{\doiorurl}{%
	\iffieldundef{url}
	{\iffieldundef{eprint}
		{}
		{http://arxiv.org/abs/\strfield{eprint}}}
	{\strfield{url}}%
}

\newcommand{\myhref}[1]{%
	\ifboolexpr{%
		test {\ifhyperref}
		and
		not test {\iftoggle{bbx:eprint}}
		and
		not test {\iftoggle{bbx:url}}
	}
	{\href{\doiorurl}{#1}}
	{#1}%
}
 % bibliography management
% hyper-references
\usepackage[
bookmarks=true,
linktocpage=true,
bookmarksnumbered=true,
breaklinks=true,
pdfstartview=FitH,
hyperfigures=false,
plainpages=false,
naturalnames=true,
colorlinks=true,
pagebackref=false,
pdfpagelabels]{hyperref}

\hypersetup{
	colorlinks,
	citecolor=blue,
	filecolor=blue,
	linkcolor=blue,
	urlcolor=blue
}

% cross-references
\usepackage[capitalize, noabbrev]{cleveref} % hyper- and cross-referencing 

% Update to MSC2020
\makeatletter
\@namedef{subjclassname@2020}{%
	\textup{2020} Mathematics Subject Classification}
\makeatother

% Table of contents
\setcounter{tocdepth}{2}
\input{aux/usualcmds}

%%%%%%%%%%%%%%%%%%%%%%
% !TEX root = ../template.tex

% Include adittional packages below
\usepackage{amsfonts}
\usepackage{graphicx}
\usepackage{xcolor}
\usepackage{enumitem}
\usepackage{caption}
\usepackage{subcaption}
\usepackage{algorithm}
\usepackage{algpseudocode}                   % add packages here
% !TEX root = ../template.tex

\newcommand{\bea}{\begin{eqnarray}}
\newcommand{\eea}{\end{eqnarray}}
\newcommand{\fixme}[1]{{{\color{blue}{[#1]}}}}
\newcommand{\AK}{\text{AK}}
\newcommand{\MS}{\text{MS}}
\newcommand{\G}{\text{G}}
\newcommand{\C}{\mathbb C}
                   % add commands here
\addbibresource{aux/bibliography.bib}  % add references here

%%%%%%%%%%%%%%%%%%%%%%
\title{Notes on the Andrews-Curtis conjecture}
% !TEX root = ../ac_paper.tex

\author{A.~Shehper}
\address{NHETC, Department of Physics and Astronomy, Rutgers University, Piscataway, New Jersey 08854, USA}

\author{A.~Medina-Mardones}
\address{Department of Mathematics, Western University, Canada}

\author{B.~Lewandowski}
\address{Institute of Mathematics, University of Warsaw, ul. Banacha 2, 02-097 Warsaw, Poland}

%\author{J.~Craven}

\author{A.~Gruen}

\author{P.~Kucharski}
\address{Institute of Mathematics, University of Warsaw, ul. Banacha 2, 02-097 Warsaw, Poland}
%\email{piotr.kucharski@mimuw.edu.pl}

\author{S.~Gukov}
\address{Richard N. Merkin Center for Pure and Applied Mathematics, California Institute of Technology, Pasadena, CA 91125, USA}


%\author[A.~Medina-Mardones]{Anibal~M.~Medina-Mardones}
%\address{Department of Mathematics, Western University, Canada}
%\email{\href{anibal.medina.mardones@uwo.ca}{anibal.medina.mardones@uwo.ca}}

\date{\today}
\subjclass[2020]{\TBW}
\keywords{\TBW}

\begin{document}
	% !TEX root = ../ac_paper.tex

\begin{abstract}
Using a long-standing conjecture from combinatorial group theory as our framework, we explore from multiple angles challenges of finding rare instances that carry disproportionately high rewards. Based on lessons learned in the mathematical context of the Andrews-Curtis conjecture, we propose algorithmic improvements that can be relevant in other domains with ultra sparse reward problems. Although our case study can be formulated as a game, its shortest winning sequences may be $10^6$ or $10^9$ times longer compared to those in a game of chess. In the process of our study, we demonstrate that one of the potential counterexamples due to Akbulut and Kirby, whose status escaped direct mathematical methods for decades, is stably AC-trivial.
\end{abstract}
	\maketitle
	% !TEX root = ../ac_paper.tex

\section{Introduction\label{sec:intro}}

This paper is structured as follows. In \autoref{sec:AC}, we review the Andrews--Curtis conjecture. In \autoref{sec:search}, we use classical search algorithms to study the presentations of Miller--Schupp series. We devise a greedy search algorithm and show that it performs significantly better than the breadth first search algorithm widely used to study this problem in the literature. We discover that the previously-known shortest potential counterexample to the stable AC conjecture, i.e. $\AK(3)$, is in fact stably AC-trivial.
\newline

In \autoref{sec:rl}, we use reinforcement learning to search through the space of balanced presentations. Specifically, we focus on the Proximal Policy Optimization algorithm. We find that while this algorithm performs significantly better than breadth first search, it does not outperform greedy search. (See \autoref{fig:performance}.)
\newline

In \autoref{sec:lm}, we use a decoder-only Transformer model to study the language structure of balanced presentations. We find that easy and hard presentations form their own clusters inside the embedding space of the Transformer model.

\begin{figure}
	\centering
	\begin{subfigure}[b]{0.5\textwidth}
		\includegraphics[width=1.1\textwidth]{fig/performance_vs_n.png}
		\caption{Distribution versus $n$}
		\label{fig:performance_vs_n}
	\end{subfigure}
	\begin{subfigure}[b]{0.5\textwidth}
		\centering
		\includegraphics[width=1.1\textwidth]{fig/performance_vs_length.png}
		\caption{Distribution versus length}
		\label{fig:performance_vs_length}
	\end{subfigure}
	\caption{The figure shows a comparison of three algorithms --- breadth first search, greedy search, and Proximal Policy Optimization (PPO) --- that we used to search through the space of balanced presentations. The number of presentations of the Miller--Schupp series, $\MS(n, w)$, solved by an algorithm is given on the vertical axis. We compare the performance as a function of $n$ (above) and the length of the presentation (below). Greedy Search consistently outperforms Breadth-First Search and Proximal Policy Optimization.}
	\label{fig:performance}\anibal{It would be better to include a PDF version of this pictures instead of a (compressed) PNG.}
\end{figure}

%We compare the performance of Greedy Search (GS), Breadth First Search and Proximal Policy Optimization on the presentations of Miller--Schupp series with $n, \ \length(w) \leq 7$ in \autoref{fig:performance}.

	% !TEX root = ../ac_paper.tex

\section{Andrews--Curtis conjecture\label{sec:AC}}

The Andrews--Curtis conjecture concerns the study of \textit{balanced presentations} of the trivial group, i.e.
presentations of the trivial group with an equal number of generators and relators.
The conjecture proposes that any balanced presentation
\[
\angles{x_1, \dots, x_n \mid r_1, \dots, r_n}
\]
can be converted to the trivial presentation
\[
\angles{x_1, \dots, x_n \mid x_1, \dots, x_n}
\]
through a series of the following operations known as \textit{AC moves} \cite{Andrews--Curtis}.
\begin{enumerate}[label=(AC\arabic*)]
	\item Substitute some $r_i$ by $r_i r_j$ for $i \neq j$.
	\item Replace some $r_i$ by $r_i^{-1}$.
	\item Change some $r_i$ to $g r_i g^{-1}$ where $g$ is a generator or its inverse.
\end{enumerate}
We will refer to the sum of the word lengths of all relators as the \textit{length} of a presentation.
Two presentations that can be transformed into each other by a sequence of AC moves are said to be \textit{AC-equivalent}.
A presentation that is AC-equivalent to the trivial presentation is referred to as \textit{AC-trivial}.
Despite considerable efforts, little progress has been made in establishing a proof of the conjecture.
However, several families of potential counterexamples have been suggested in the literature.

To investigate a given presentation, one may systematically explore the entire space of possible sequences of AC moves in search of a sequence that renders the presentation trivial.
This space grows exponentially with the length of the sequence.
For a presentation with $n$ generators, there are $3n^2$ AC moves, and the total number of sequences of AC moves of length $k$ is $(3n^2)^k$.
Even for a modest case like $n=2$ and $k=20$, the number of possible sequences is on the order of $10^{21}$, making a brute-force approach impractical.


Classical search algorithms such as genetic algorithms \cite{genetic}, and breadth-first search \cite{bfs-ac} have been employed to search through this space and achieved success in trivializing balanced presentations with two generators and lengths less than 13.
The following presentation of length 13,
\[
\angles{x, y \mid x^3 = y^4, xyx = yxy}
\]
is the shortest presentation, up to AC-equivalence, that eludes all attempts at length reduction.
This presentation is a part of an infinite series of potential counterexamples by Akbulut and Kirby \cite{Akbulut--Kirby}:
\[
\AK(n) = \angles{x, y \mid x^n = y^{n+1}, xyx = yxy}, \quad n \geq 2.
\]
$\AK(2)$ has length 11 and has been established as AC-trivial \cite{genetic} whereas $\AK(3)$ is the aforementioned presentation with length 13.


In over two decades since the first utilization of search algorithms \cite{genetic, bfs-ac}, only unsuccessful attempts have been made to trivialize $\AK(3)$ with different variants of breadth-first search algorithm using an increased amount of computational resources \cite{Bowman-McCaul, krawiec2016distance, Panteleev-Ushakov}.
Notably, \cite{Panteleev-Ushakov} found that no sequence of AC moves that allows relator lengths to increase up to 20 trivializes $\AK(3)$.
This lack of success could be interpreted as suggestive evidence that $\AK(3)$ might be a counterexample to the Andrews--Curtis conjecture.
However, recent works by Bridson and Lishak have shown that there exist AC-trivializable balanced presentations of the trivial group, for which the number of AC moves in a trivializing sequence is bounded below by a superexponential function of the length of the presentation \cite{Bridson, Lishak}.
Roughly speaking, for these presentations, if the sum of word lengths is $k$, the number of AC moves required to trivialize the presentation is at least $\Delta (\lfloor \log_2 k \rfloor)$ where $\Delta \colon \mathbb{N} \to \mathbb{N}$ is defined recursively as $\Delta(0) = 2$ and $\Delta (j) = 2^{\Delta(j-1)}$ for $j \geq 1$.
In particular, $\Delta (\lfloor \log_2 (13) \rfloor) = 65536$, whereas presentations trivialized by the aforementioned search algorithms have AC sequences of length less than $1000$.
While $\AK(3)$ is itself not a member of the family of examples studied by Boris and Lishak, their findings challenge the inclination to view it as a counterexample.
Their work also underscores the necessity of employing search methods that are more efficient than breadth-first search.


In this paper, we will consider a variety of computational tools to better understand the properties of balanced presentations of the trivial group.
We will test the efficacy of our approaches on a subset of presentations from the Miller--Schupp series of potential counterexamples \cite{Miller--Schupp}:
\[
\MS(n, w) = \angles{x, y \mid x^{-1} y^n x = y^{n+1}, x = w}.
\]
Here, $n > 0$, and $w$ is a word in $x$ and $y$ with zero exponent sum on $x$.
For $w_\star = y^{-1} x^{-1} y x y$, the presentations $\MS(n, w_\star)$ are AC-equivalent to the presentations from Akbulut--Kirby series \cite{MMS}.
In particular, the presentation
\[
\MS(n, w) = \angles{x, y \mid x^{-1} y^3 x = y^{4}, x =  y^{-1} x^{-1} y x y}.
\]
of length 15 is AC-equivalent to $\AK(3)$.

We will only consider presentations with $n,\, \length(w) \leq 7$.
Our selection criteria aimed to strike a balance: we sought a dataset of presentations large enough to allow for meaningful analysis, yet small enough to ensure all computations are feasible within a practical timeframe.
Additionally, we reduced $x^{-1}w$ freely and cyclically and kept only one representative of each orbit under the action of cyclic permutations.\anibal{What is the action?}
After these simplifications, we were left with $170$ choices for $x^{-1} w$.
This resulted in a dataset of $7 \times 170 = 1190$ presentations from the Miller--Schupp series.


Our implementation of AC transformations differed from the AC transformations mentioned above in two ways.
First, we considered the following set of operations.
\begin{enumerate}[label=(AC$'$\arabic*)]
	\item Substitute some $r_i$ by $r_i r_j^{\pm 1}$ for $i \neq j$.
	\item Change some $r_i$ to $g r_i g^{-1}$ where $g$ is a generator or its inverse.
\end{enumerate}
For two generators, which is the only case we study in this paper, the group generated by these AC transformations is isomorphic to the group generated by the original AC transformations.\footnote{
The difference lies in how the inversion of a relator is handled: we always follow an inversion by a concatenation, while the original AC moves allow for standalone inversion moves.
The original inversion moves may be retrieved from the new generators as follows.
For a given presentation $\angles{x_1, x_2 \mid r_1, r_2}$, the sequence of moves: $r_2 \to r_2 r_1$, $r_1 \to r_1 r_2^{-1}$, $r_2 \to r_2 r_1$, and $r_2 \to r_1 r_2 r_1^{-1}$ results in the presentation $\angles{x_1, x_2 \mid r_2^{-1}, r_1}$, which is the same as $r_2 \to r_2^{-1}$ up to swapping the two relators.
We also enhanced the notion of trivial presentation(s) to include all presentations of length 2: $\{\angles{x_1, x_2 \mid x_i^{a}, x_j^{b}}  \mid i, j = 1, 2; a, b = \pm 1; i \neq j \}$.
}
The reason for this change is due to its effect on performance in greedy search and reinforcement learning algorithms studied in \autoref{sec:search} and \autoref{sec:rl}.
Specifically, the length of a presentation provides a useful signal when searching through the space of presentations with these algorithms.
An inversion transformation leaves the length invariant providing no signal to the search process and slowing down the performance of the algorithm significantly.
For the rest of the paper we will refer to the new transformations (instead of the original AC transformations) as ``AC transformations" or ``AC moves".

Second, in order to make the search space finite in size, we set a maximum length that each relator is allowed to take.
If an AC transformation resulted in a presentation with a relator of length greater than this maximum length, the AC transformation was set to act trivially.
In the search of a sequence of AC moves that trivialize a presentations of the Miller--Schupp series $\MS(n, w)$, we set this maximum length to be $2 \times \text{max}(2 n+3, \length(w)+1) + 2$.
This specific choice was made to allow for at least one concatenation move followed by a conjugation move in the search process.
%Note that this constraint is quite restrictive: any presentation of the Miller--Schupp series that requires the length of a relator to grow to more than the set maximum length would not be trivialized.
	% !TEX root = ../ac_paper.tex

\section{Language Modeling} \label{sec:lm}

In this section, we discuss a model for the ``language" of balanced presentations.
Each presentation with two relators is a sequence made of six letters, also known as ``tokens" in the nomenclature of Natural Language Processing, i.e. $x$, $y$, $x^{-1}$, and $y^{-1}$, and two ``stop tokens": one that separates two relators of a presentation and another that marks the end of a presentation.
Given this vocabulary $V$ of six tokens, we can ask what is the probability $p(t_1, \dots, t_N)$ for $t_i \in V$ of the occurrence of a specific presentation in the space of all balanced presentations.
Using the chain rule of probability theory,
\[
p(t_1 \cdots t_{N}) = \prod \limits_{i=1}^{N} p (t_{i} \mid t_{1} \cdots t_{i-1})
\]
Here $p (t_{N} \mid t_{1} \cdots t_{N-1})$, often called the $N$-gram probability distribution, is the probability of a token $t_N$ following a sequence of tokens $t_{1} \cdots t_{N-1}$.
To model the language of balanced presentations, we can alternatively estimate the $N$-gram probability distributions for all $N$.

Over the last few years, Transformer models have shown great success in modeling human-understandable languages to the extent that these models can create text almost indistinguishable from that of a human expert.
Specifically, the architecture used for modeling language is the auto-regressive ``decoder-only" Transformer, which we review in detail in \autoref{sec:transformer_review}.
In \autoref{sec:transformer_datasets}, we discuss the method with which we generate the dataset required for training the model.
Finally, in \autoref{sec:transformer_results}, we share details of some insights we learned from this process.

\subsection{Transformers: a review\label{sec:transformer_review}}

Here, we give a short review of the architecture of a decoder-only transformer.
For more details, see \cite{vaswani2023attention, elhage2021mathematical, douglas2023large}.

Given an input sequence $t_1, t_2, \dots, t_{N}$, a decoder-only transformer predicts the probability distribution $p(t \mid t_1, t_2, \dots, t_{N})$ over the set $V$ of tokens of size $n_{\text{vocab}}$.
The probability is computed by applying the softmax function to the logits $T(t)$, which are estimated by applying the following sequence of operations.\footnote{
The softmax function, $\softmax \colon \R^n \to (0, 1)^{n}$, is defined as $\softmax(x)_i = e^{x_i} / \sum\limits_{j=1}^n e^{x_j}$.}
First, assign to each token in the vocabulary a distinct label in the range $1, 2, \dots, n_{\text{vocab}}$; re-writing the original sequence as a sequence of integers.
We will label these integers also as $t_i$.
Next, write the sequence in terms of ``one-hot encoded vectors", i.e. a matrix $t \in \R^{N \times n_{\text{vocab}}}$ such that
\[
t_{ij} = \delta_{i t_i}
\]
and embed the sequence in a $\dm$-dimensional vector space,\footnote{
Here, $t$ and all $x_j$ are two-dimensional tensors.
Hence, it is appropriate to apply tensors of linear transformations to them.
Often in a transformer architecture, these operations are of the form $\mathbb{1} \otimes \cdots$; in these cases, we drop the identity transformation and simply write the operation as $\cdots$.
For example, $\mathbb{1} \otimes W_U$, $\mathbb{1} \otimes W^m_I$, $\mathbb{1} \otimes W^m_O$, etc.
In this case, we will sometimes write $W_U$, $W^m_I$, $W^m_O$ respectively, assuming it is clear from the context and the dimensionality of these matrices that they are tensored with identity transformations.}
\[
x_0 = (W_P \otimes \mathbb{1} + \mathbb{1} \otimes W_E) t
\]
Here, $W_P \in \R^{\dm \times N}$ and $W_E \in \R^{\dm \times n_{\text{vocab}} }$ are matrices of learnable parameters, known as the ``positional embedding" and ``token embedding" matrices.

An $L$-layer transformer alternates between applying a ``multi-head attention layer" ($\sum\limits_{h \in H} h$) and an ``MLP-layer" ($m$) $L$ times.
For $i=0, \dots, L-1$,
\[
\begin{aligned}
	x_{2i+1} &= x_{2i} + \sum_{h \in H} h(\LN(x_{2i})), \\
	x_{2i + 2} &= x_{2i + 1} + m(\LN(x_{2i + 1})).
\end{aligned}
\]
Each $x_j$ is an element of $\mathbb{R^{N \times \dm}}$, with the interpretation that its $i$-th row is the embedding of the sequence $t_1, \dots, t_i$ in the embedding space $\R^{\dm}$ as learned by the preceeding $j+1$ operations.
Finally, one applies an ``unembedding layer", $W_U \in \R^{n_{\text{vocab}} \times \dm}$, to convert the output of the final layer to an $n_{\text{vocab}}$-dimensional vector of logits that estimate the sought-after probability distribution.
\[
\begin{aligned}
	T(t) &= W_U x_{2L-1}, \\
	p(t) &=\softmax (T(t)).
\end{aligned}
\]

The functions $\LN$, $m$ and $h$ are defined as follows.
$\LN$ is the LayerNorm operation that normalizes the input of each layer to make the optimization process more stable~(\cite{ba2016layer}):
\[
\LN(x) = \left(\mathbb{1} \otimes \text{diag}(\gamma) \right) \frac{(x-\overline{x})}{\sqrt{\text{var}(x)}} + \mathbb{1} \otimes \beta\,.
\]
Here, $\overline{x}$ and $\text{var}(x)$ are mean and variance of each row of $x$, and $\gamma, \beta \in \R^{\dm}$ are learnable parameters.
The MLP-layer $m$ is a non-linear operation,
\[
m(x) =W ^m_O \ \max(W_I^m x, 0)
\]
with learnable parameters $W^m_I \in \R^{d_{\text{MLP}} \times \dm}$, $W^m_O \in \R^{\dm \times d_{\text{MLP}}}$.
It is standard to set $d_{\text{MLP}} = 4 \dm$.

Finally, the multi-headed attention-layer $\sum_{h \in H} h$ is a sum of $n_{\text{heads}}$ ``attention-head" operations $h$, where
\[
h(x) = (A^h(x) \otimes W^h_O W^h_V) x.
\]
Here, $W^h_V \in \R^{d_{\text{head}} \times \dm}$,
$W^h_O \in \R^{\dm \times d_{\text{head}}}$
are matrices of learnable parameters;
$d_{\text{head}}$ is the ``attention-head dimension" that satisfies $d_{\text{head}} \times n_{\text{head}} = d_{\text{model}}$; and the attention matrix $A^h$ is computed with the help of learnable matrices
$W^h_Q, W^h_K \in \R^{d_{\text{head}} \times \dm}$,
\[
A^h(x) = \softmax^\star \left(\frac{x^T (W^h_Q)^T W^h_K x}{\sqrt{d_{\text{head}}}}\right).
\]
The attention-head is an $N \times N$ matrix, with the interpretation that $A^h(x)_{ij}$
is the ``attention" paid to the token
$t_j$ in estimating
$p(t_{i+1} \mid t_1, \dots, t_i)$.
$\softmax^\star$ is a variant of the $\softmax$ function suitable for auto-regressive tasks: it sets the upper triangular part of its input to zeros before applying the
$\softmax$ operation.
That is, future tokens, $t_k$ for $k > i$, play no role in the prediction of $p(t_{i+1} \mid t_1, \dots, t_i)$.

We train the transformer model by minimizing the cross-entropy loss between the distributions of predicted and correct labels for the next tokens in a sequence.
The parallelism offered by the processing of all tokens in a sequence at once is extremely beneficial for efficient training of the model for the language modeling task.

In practice, the embedding matrix $W_E$ and the unembedding matrix $W_U$ are often ``tied" together, i.e. $W_E = W_U^T$ \cite{press2017using, inan2017tying}.
The rows of $W_E = W_U^T$ are interpreted as the embeddings of words/sentences, to which one may apply the usual operations of a vector space \cite{Bengio:2003, mikolov-etal-2013-linguistic}.
For example, the cosine of the angle between two embedding vectors, also known as the ``cosine similarity", is often used to measure the similarity between two texts.
Two semantically similar texts have higher cosine similarity between them, while semantically different texts correspond to (almost) orthogonal vectors in the embedding space.

\subsection{Training and Evaluation Datasets\label{sec:transformer_datasets}}

We now discuss the training and validation datasets used to train and evaluate our Transformer model.
As our main interest in this paper has been in the presentations of the Miller--Schupp series, we generated a dataset of balanced presentations that are AC-equivalent to the Miller--Schupp presentations.
Specifically, we apply sequences of AC-moves to the 1190 presentations with $n \leq 7$ and $\length(w) \leq 7$ discussed in \autoref{sec:AC}, creating a dataset of about 1.8 million presentations.
Approximately 1 million of these presentations are AC-equivalent to the presentations that remained unsolved by greedy search (c.f. \autoref{sec:search}).
Only a small amount (roughly 15 percent) of the original Miller--Schupp presentations were part of this dataset.

The dataset is tokenized using six tokens: two stop tokens and one token each for the two generators and their inverses.
The tokenized dataset had about $2.17 \times 10^8$ tokens.
As our goal is to get insights into properties that distinguish GS-solved and GS-unsolved presentations, we performed an exploratory data analysis of the two subsets of data associated to these presentations.
We plot the percentage of appearance of each token for these subsets in \autoref{fig:tokens_hist}.
The ratio of frequency of $y^{\pm 1}$ to the frequency of $x^{\pm 1}$ is higher in the GS-unsolved dataset.
This is likely because the GS-unsolved presentations have larger $n$, and larger $n$ corresponds to a higher number of occurrence of $y^{\pm 1}$ in the Miller--Schupp presentation.
Interestingly, this effect remains in the dataset even after applying thousands of AC-moves to the original presentations.

We paid special attention to ensure that our dataset contains presentations of a wide range of lengths so as not to bias our model towards learning trends specific to any fixed length.
To this end, we devised an algorithm (\autoref{alg:apply_ac_moves} in \autoref{app:algorithm}) that creates an almost uniform distribution over the lengths of the presentations.
(See \autoref{fig:gpt_data}.) We set aside $10\%$ of our entire data for validation.

\begin{figure}
	\centering
	\includegraphics[scale=0.15]{fig/tokens_hist.pdf}
	\caption{Percentage of appearance of each token in the two subsets of the training dataset that are equivalent to GS-solved and GS-unsolved presentations.
	To be clear, we computed the percentages separately for each subset of the training data, i.e. the heights of all blue (and orange) bars adds separately to 100.}
	\label{fig:tokens_hist}
\end{figure}

\begin{figure}
	\centering
	\includegraphics[scale=0.15]{fig/gpt_data_length_distribution.pdf}
	\caption{Percentage of presentations in various ranges of lengths.
	Percentages were computed independently for the two subsets of the dataset, corresponding to presentations that are AC-equivalent to GS-solved and GS-unsolved presentations.
	We used \autoref{alg:apply_ac_moves} from \autoref{app:algorithm} to ensure the almost-uniform distribution depicted here.}
	\label{fig:gpt_data}
\end{figure}

\subsection{Results}\label{sec:transformer_results}

A randomly initialized model with the initialization scheme given in \cite{Radford2019LanguageMA} has a cross entropy loss of
%$- \ln \left(\frac{1}{n_{\text{vocab}}} \right) \approx - 1.7918$,
$-\ln (1/n_{\text{vocab}}) \approx 1.7918$.
With training, we could achieve a validation loss of $0.7337$.\footnote{
We tuned the hyperparameters a little but it is quite likely that one can achieve a better performing model with more hyperparameter tuning.
Similarly, more training data will necessarily help with the performance.
We trained a Transformer model with hyperparameters given in \autoref{app:hyperparameters}.}
We used the untrained and the trained model to get the embeddings of all $1190$ presentations of the Miller--Schupp series with $n \leq 7$ and $\length(w) \leq 7$.
We used t-SNE to project these embedding vectors to a plane \cite{JMLR:v9:vandermaaten08a}.
The plots are shown in grid in \autoref{fig:tsne_embeddings}.

Each row of \autoref{fig:tsne_embeddings} corresponds to a fixed value of $n$.
The left (resp. right) column depicts t-SNE projections of embeddings obtained by an untrained (resp. trained) model.
t-SNE dependence on a distance measure: it learns to map vectors that are closer together in the higher-dimensional space, with respect to this distance measure, close together in the plane \cite{JMLR:v9:vandermaaten08a}.
We used cosine simiarity between embedding vectors as the distance measure for our plots.
We note that the GS-solved and GS-unsolved presentations seem to cluster much more in the plots in the right column.
This indicates that a trained Transformer model is able to distinguish between GS-solved and GS-unsolved presentations to a good extent, albeit not perfectly.\footnote{
Note also that t-SNE admits a hyperparameter known as ``perplexity", and the projections learned by t-SNE depend on the hyperparameter \cite{wattenberg2016how}.
Thus, in general, t-SNE plots must be interpreted with care.
The plots shown in \autoref{fig:tsne_embeddings} were all made with the perplexity value of $30$.
We checked however that the clusters of GS-solved and GS-unsolved presentations continue to exists at a broad range of perplexity values.}

Note that the training dataset contained no information about the ease of solvability of a presentation.
It also did not contain many presentations of the Miller--Schupp series itself.
Instead, it contained presentations that are AC-equivalent to the Miller--Schupp series presentations.
Our observation that a Transformer model trained on this dataset can distinguish between the GS-solved and GS-unsolved presentations indicates that:
\begin{enumerate}[label=\alph*)]
	\item There likely exists an invariant at the level of the ``language`` of the balanced presentations that distinguishes GS-solved vs GS-unsolved presentations.
	\item This invariant survives application of thousands of AC-moves we used to generate the training examples in our dataset.
\end{enumerate}

\begin{figure}
	\centering
	\includegraphics[scale=0.16]{fig/embeddings.pdf}
	\captionsetup{width=1.1\textwidth}
	\caption{t-SNE plots depicting embeddings of the Miller--Schupp presentations $MS(n, w)$.
	The left (right) column shows embeddings learned by an untrained (trained) transformer model.
	Each row corresponds to a value of $n$.
	Trained models cluster together GS-solved and GS-unsolved presentations indicating the possibility of a difference in the linguistic structure of the two sets of presentations.}
	\label{fig:tsne_embeddings}
\end{figure}

	% !TEX root = ../ac_paper.tex

\section{Reinforcement Learning}

While greedy search algorithm performs better than breadth first search, it has some of the same downsides. Namely, it is memory inefficient, and we cannot leverage the parallelizability of modern hardware architectures. A simple candidate for algorithms that do not have these downsides are reinforcement learning algorithms. In particular, the policy gradient algorithms, which we will review in section 4.1 are memory efficient and can be trained in a highly distributed manner. We will note that despite the use of only a fraction of compute generally available for research purposes, policy gradient algorithms seem to do well. In particular, they perform better than breadth first search algorithm in terms of the number of presentations of the Miller-Schupp series that they are able to solve, and they are able to give shorter sequences of AC moves compared to the greedy search algorithm in cases where they solve a presentation. 

This section is divided as follows: in subsection 4.1, we discuss how the problem underlying Andrews-Curtis conjecture can be modelled as a Markov Decision Process. In subsection 4.2, we will discuss some details of a specific reinforcement learning algorithm, called Proximal Policy Optimization algorithm that we used to find sequences of AC moves. Finally, in subsection 4.3, we discuss the results of our work, comparing the performance of PPO with that of the classical search algorithms studied in the previous section. 

\subsection{Markov Decision Process}

A Markov Decision Process is a 5-tuple $(S, A, R, P, \rho)$ where 
\begin{itemize}
	\item $S$ is the space of states, 
	\item $A$ is a set of actions, i.e. $a \colon S \to S \ \forall \ a \in A$, 
	\item $R \colon S \times A \times S \to \mathbb{R}$ is the ``reward" function, 
	\item $P \colon S \times A \to \mathcal{P}(S)$ is the transition probability function, and 
	\item $\rho$ is the initial probability distribution of states. 
\end{itemize}

The schematic picture of how these objects interact with each other is as follows. We start with a state $s_0$ sampled from the distribution $\rho$ and take an action $a_0$. This results in a state $s_1$ with probability $P(s_1 \mid s_0, a_0) $. The transition gets a ``reward" $r_0 = R(s_0, a_0, s_1)$ which quantifies the effectiveness of the action in contributing toward achieving an ultimate goal. From state $s_1$, we repeat this process, obtaining a trajectory of states
\[
\tau = \left( s_0, a_0, s_1, a_1, \cdots \right)
\]
The goal of this process is to maximize the cumulative return,
\[
R(\tau) = \sum\limits_{t=0}^{T} \gamma^t R(s_t, a_t, s_{t+1})
\]
Here, $T$ is the length of the trajectory and $\gamma \in \left(0, 1 \right)$ is the ``discount factor" that assigns smaller weight to the reward values obtained in the future. 
\newline

For the problem under investigation in this paper, i.e. finding a sequence of AC transformations that trivialize a balanced presentations, 


In the case of Andrews-Curtis conjecture, $S$ is the space of all balanced presentations with two generators, and $A$ is the set of AC transformations. We start with a presentation, of say Miller-Schupp series, and apply  Using a deep learning algorithm, we hope to learn an optimal transition probability function such that 
$P$ is the transition probability function 

We will model the problem underlying the Andrews-Curtis conjecture as a Markov Decision Process (MDP).
An MDP is a tuple of data $(S, A, P, R, \rho)$ where
\begin{itemize}
	\item $S$ is a set of states called the state space.
	\item $A$ is a set of actions that acts on $S$, called the action space.
	That is, each element $a \in A$ is a map $a \colon S \to S$.
	\item $P \colon S \times A \times S \to \mathbb{R}$ is the transition
	probability function.
	It is the probability $p (s' | a, s)$ of reaching
	state $s'$ as we take action $a$ in state $s$.
	\item $R \colon S \times A \times S \to \mathbb{R}$ is the 'reward' function.
	\item $\rho$ is the initial probability distribution of states.
\end{itemize}

In the setting of the Andrews-Curtis conjecture, $S$ is the set of all presentations of the trivial group, and $A$ is the set of AC moves.
The transition probability function $P$ is the probability of applying a particular AC move to a given presentation.
The choice of reward function $R$ is up to us.
One suitable option (the only one we have worked with so far) is to take $R(s)$ for a presentation $s$ to be the negative of the total word length ---i.e., the sum of word lengths of each relator in the presentation.
(AK(3) has a total word length of 13.)
This choice depends only on the final state $s'$ and is independent of the initial state $s$
or the action $a$.
\footnote{Another good choice could be the difference in total word lengths of the final and the initial states.} It is suitable for the goal of trivializing a presentation as the trivial presentation has the maximum possible reward value: negative of the number of generators.

Differences in our AC moves and the usual notion of AC moves: 1.
since we restrict lengths, our AC moves are not invertible; 2.
we use full-simplify which cyclically reduces words before returning the answer.
For example, if we conjugate a word and it takes the form $y \cdots y^{-1}$, our conjugation operation returns $\cdots$.


\subsection{Proximal Policy Optimization}


\subsection{Results}

\begin{enumerate}
\item Shorter sequences of AC moves. 
\end{enumerate}

	% !TEX root = ../ac_paper.tex

\section{Results and current challenges}

Why RL and not BFS? 1. memory efficient, 2. parallelizable/distributed. As a consequence,  we can achieve higher bounds on the length of sequence of AC moves that we can apply on a presentation. Figure 1 in Bowman-McCaul seems to show maximum depth to be around 40. That is, at best, we can search through sequences of 40 AC moves with BFS.

What is the limit on maximum path length? What's the limit achieved by previous BFS papers?
\subsection{Potential counterexamples of AC}
\subsubsection{Miller-Schupp series}
With the help of Breadth-first search and PPO, we have been able to show that all examples of Miller-Schupp series of total word length 14 are equivalent to either trivial presentation or $\AK(3)$, with the following two exceptions:
\bea
\langle x, y | x^{-1} y^2 x = y^{3} , x = y x^2 y^{\pm 1} x^{-2}\rangle
\eea
It is listed as example 1 on page 3 of \cite{MMS}. I have checked whether we can reduce the length of these examples with max length = 20.

\fixme{Compare with Table 2 of Panteleev-Ushakov. They seem to have a lot more examples with $n=2$ that they could not resolve.}

\subsubsection{Gordon's series}


\subsection{Does stable AC equivalence imply AC equivalence?}

In \cite{MMS}, the authors give examples of stably AC-equivalent pairs of presentations for which it is not known whether they are also AC-equivalent. We consider one of these examples, which is stably AC-equivalent to the trivial presentation:
\bea
<x, y | xyx^{-2}y^{-1} xy^{-1}, x^{-1} y^{-1} x y^2 x y^{-1}>.
\eea

We have shown that it is AC-equivalent to $\AK(3)$. This implies that $\AK(3)$ is stably AC-equivalent to the trivial presentation.

\fixme{This is AC-equivalent to $[ 2  2 -1 -1 -1  2  2  0  0  0  0  0  0  0  0  0  0  0  0  0 -1 -2 -1  2
	1  2  0  0  0  0  0  0  0  0  0  0  0  0  0  0]$ as shown by $mms_to_ak3.py$ around update 99.

	$x^2 y^{-3} x^2 = 1 \to x^2 y^3 = x^{-2} \to x^{4} y^3 =1 \to y^3 x^{-4} =1 \to y^3 = x^4 $ and $x^{-1} y^{-1} x^{-1} y x y = 1 \to x y x = yxy$. }

We can relate the length-25 example of MMS to AK(3) in 53 moves: $[(9, 25), (7, 25), (4, 17), (8, 17), (11, 17), (5, 17), (11, 17), (9, 17), (3, 16), (10, 16), (12, 16), (7, 16), (7, 16), (9, 16), (11, 16), (5, 16), (3, 15), (5, 15), (4, 19), (3, 14), (12, 14), (5, 14), (7, 16), (7, 18), (1, 19), (9, 19), (11, 19), (8, 19), (3, 18), (5, 18), (10, 18), (2, 15), (6, 15), (12, 15), (9, 15), (7, 15), (5, 15), (11, 15), (10, 15), (3, 15), (8, 15), (11, 15), (9, 15), (2, 16), (10, 16), (12, 16), (5, 16), (7, 16), (9, 16), (11, 16), (1, 13), (9, 13), (8, 13)]$.

The best BFS implementation that works is in src/breadth-first-search/bfs.py.

I need to change -- to lengths in ACMove in my implementations of ACMove.

This also implies that the simplest example of a presentation that satisfies the weak AC conjecture but not the AC conjecture is AK(3).

It would be nice if we can also solve AK(4). Even if it happens through a computer program involving BFS.

I should look at the results of my attempt to trivialize MMS-proposition-1.2 example on W\&B.

This is one of those examples that satisfies the weak AC conjecture but was not known to satisfy the AC conjecture. If it is equivalent to trivial instead of AK(3), we have obtained a new result. One question though: in the paper, they obtain this presentation from a rank-3 presentation after substitution. But is substitution a valid AC move (or a sequence of it)?

Note that Theorem 1.6 of MMS says that there exist stably AC-inequivalent but AC-equivalent 2-element sets.

" It seems to
be another very difficult question as to whether or not there are stably ACequivalent, but AC-inequivalent sets of the same cardinality in a free group." [MMS]

	% !TEX root = ../ac_paper.tex

\subsection*{Acknowledgment}

We would like to thank Anna Beliakova, Michael Douglas, Konstantin Korovin, Alexei Lisitsa, Maksymilian Manko, Ciprian Manolescu, Fabian Ruehle, Josef Urban, and Tony Yue Yu for insightful discussions and comments. We especially want to thank Anna Beliakova for igniting our interest in the Andrews--Curtis conjecture as a framework for exploring problems with long and rare sequences of moves that an RL agent must discover.

The work of A.S. is supported by the US Department of Energy grant DE-SC0010008 to Rutgers University. The authors acknowledge the contributions of Office of Advanced Research Computing (OARC) at Rutgers University for providing access to the Amarel cluster and other computing resources.
A.M.'s work is supported by NSERC grants RES000678 and R7444A03. A.M. also gratefully acknowledges the excellent working conditions provided by the Max Planck Institute for Mathematics in Bonn.
The work of P.K. and B.L. is supported by the SONATA grant no. 2022/47/D/ST2/02058 funded by the Polish National Science Centre. This research was carried out with the support of the Interdisciplinary Centre for Mathematical and Computational Modelling at the University of Warsaw (ICM UW).
The work of S.G. is supported in part by a Simons Collaboration Grant on New Structures in Low-Dimensional Topology, by the NSF grant DMS-2245099, and by the U.S. Department of Energy, Office of Science, Office of High Energy Physics, under Award No. DE-SC0011632.
	\sloppy
	\printbibliography
\end{document}

% Outline:
% 1. Andrews-Curtis conjecture
% 2. Reinforcement Learning
% 3. Progress and current challenges
% 4.

% Successes:
% 1. We are able to find the shortest sequence of AC moves that trivializes AK(2) with maximum length = 7.

% Main troubles:

% 1. Say we have solved AK(2) for maximum length = 7 (which we have). We want to solve it again after setting maximum length = 20. Now we have to do hyperparameter search again, and there is no guarantee that we will find a good set of hyperparameters in the same range that we used while conducting hyperparamter search for (AK(2), 7). This is a time consuming and resource intensive process with possibly requiring thousands of experiments to be run.

% Possible solutions:
% a). Based on the hyperparameters for the best run of (AK(2), 7), there is a (possibly clever) set of hyperparameters for which (AK(2), 20) will be near optimal.
% b). So far, I have only used random search. Maybe using a more sophisticated search approach will cut down time. I think Ray Tune has some sophisticated search algorithms that 'stop bad runs early, save intermediate results, restart from previous trials, or pause and resume runs'.
% b). Use maximum update parameterization.

% 2. Policy gradient algorithms can learn to do better if they solve the problem in the first place. With very large maximum length such as

% Possible directions:
% 1. In addition to AK(3) (which has been focus of our work so far), try running experiments for presentations of the other family (the one that's parameterized by w). We have a list of these presentations up to maximum length=20,

% Some ideas to try if our current approach does not work out:
% 1.

% Some mathematical ideas to try if our current approach does not work out:

% 1. It might be useful to try using Whitehead