% !TEX root = ../ac_paper.tex

\section{Isolated components and isolation values}\label{s:isolation}

Let's recall that in the AC conjecture, the focus is on $k$-balanced presentations and their AC simplifications.
An AC simplification of a presentation $\pi$ is a sequence $(\pi_0, \dots, \pi)$ of presentations, with $\pi_0$ being the trivial presentation, such that all consecutive pairs in the sequence are related by an AC move.
Since $\pi_0$ minimizes length among all balanced presentations, one might adopt a greedy strategy when attempting to simplify a given $\pi$: given $\pi_k$, select $\pi_{k-1}$ such that the length of $\pi_{k-1}$ is either reduced or, at the very least, not increased relative to $\pi_k$.

While this approach proves useful in numerous instances, it is not universally applicable.
A more nuanced perspective involves considering simplifications $(\pi_0, \dots, \pi_m)$ where the length of each $\pi_k$ does not exceed that of $\pi$.
Our focus will be on this scenario, referring to presentations that resist such simplifications as \textit{isolated}.
To quantify the degree of isolation, we introduce an \textit{isolation value} for each presentation, an integer that in extreme cases is $0$ for presentations that are not isolated and $\infty$ for counterexamples to the AC conjecture, if any exist.
Since isolated presentations are not necessarily isolated from each other, we introduce an equivalence relation that allows us to better count these isolated components and their isolation values.

Employing methodologies from topological data analysis, our objective in this section is to empirically count isolated components and compute their isolation values in increasing neighborhoods of the identity presentation.

\subsection{Formalization}

We provide the basic definitions and conceptual basis for the computations presented in the next subsection.

\subsubsection{Based graphs}

A \textit{based graph} is a pair $(\Gamma, v_0)$ consisting of a graph $\Gamma$ and a preferred vertex $v_0$ in it.
We will often drop $v_0$ from the notation.
A \textit{based subgraph} $\Gamma_n$ of $\Gamma$, written $\Gamma_n \leq \Gamma$, is a subgraph $\Gamma_n$ of $\Gamma$ with the same preferred vertex.
We say that $\Gamma_n$ is \textit{full} in $\Gamma$ if for any two vertices in $\Gamma_n$ joined by an edge in $\Gamma$, the edge is also in $\Gamma_n$.
A \textit{filtration} of a based graph $\Gamma$ is a collection
\[
\Gamma_0 \leq \Gamma_1 \leq \Gamma_2 \leq \dotsb
\]
of based subgraphs of $\Gamma$ for which each vertex and edge of $\Gamma$ is in $\Gamma_n$ for some $n$.
We refer to $\Gamma$ as a \textit{filtered based graph}.
If each $\Gamma_n$ is full in $\Gamma$ we refer to the filtration as full and notice that full filtrations are equivalent to $\N$ valued functions from the set of vertices of $\Gamma$ sending $v_0$ to $0$.

\subsubsection{The AC graph}

Let $\Gamma^{\text{AC(k)}}$ be the graph whose vertices are $k$-balanced presentations, based at the trivial presentation, having an edge between two vertices if there is an AC move between them.
Additionally, $\Gamma^{\text{AC(k)}}$ is equipped with a full filtration obtained from the function sending a vertex to the length of its presentation minus $k$.

\subsubsection{Filtration and connectivity values}

Given a filtered based graph $(\Gamma, v_0)$.
The \textit{filtration value} $\Filt(v)$ of a vertex $v$ is the smallest $n \in \N$ such that $v$ is a vertex in $\Gamma_n$.
Similarly, its \textit{connectivity value} $\Conn(v)$ is the smallest $n \in \N$ such that $v$ and $v_0$ can be joined by a path in $\Gamma_n$ or is set to $\infty$ if such path does not exist in $\Gamma$.

\subsubsection{Isolation value and isolation components}

The \textit{isolation value} of a vertex $v$ in a filtered based graph is defined as
\[
\Isol(v) = \Conn(v) - \Filt(v),
\]
a number in $\N \cup \set{\infty}$.
A vertex is said to be \textit{isolated} if its isolation value is positive.

We introduce an equivalence relation on isolated vertices.
Two belong to the same \textit{isolated component} if they have the same connectivity value, say $n$, and they can be joined by a path in $\Gamma_{n-1}$.
The \textit{isolation value} of a component is the maximum of the isolation values of its elements.

\subsection{Results}

Let $\Gamma(\ell)$ be the full subgraph of $\Gamma^{\text{AC(2)}}$, with the induced filtration, whose vertices are presentations that can be joined to the trivial one via paths that contains only presentations of length at most~$\ell$.
Naturally, for any $\ell$ and $n$ we have that $\Gamma(\ell)_n$ is a full subgraph of $\Gamma(\ell+1)_n$.
We will denote by $v(\ell)$ and $e(\ell)$ the number of vertices and edges of $\Gamma(\ell)$.
Let us denote by $ic(\ell)_k$ the number of isolated components with isolation value $k$.
Tables \ref{fig:classical_persistence} and \ref{fig:prime_persistence} summarize our results.

\begin{figure}
	\begin{tabular}{|c|c|c|c|c|}
		\hline
		$\ell$ & $v(\ell)$ & $e(\ell)$ & $ic(\ell,1)$ & $ic(\ell,2)$ \\ \hline
		3 & 36 & 72 & 0 & 0 \\ \hline
		4 & 100 & 248 & 0 & 0 \\ \hline
		5 & 388 & 1072 & 0 & 0 \\ \hline
		6 & 884 & 2376 & 0 & 0 \\ \hline
		7 & 3892 & 10775 & 0 & 0 \\ \hline
		8 & 9172 & 25675 & 0 & 0 \\ \hline
		9 & 37428 & 106513 & 0 & 0 \\ \hline
		10 & 84996 & 239733 & 0 & 0 \\ \hline
		11 & 350356 & 1002439 & 4 & 0 \\ \hline
		12 & 791140 & 2251375 & 16 & 0 \\ \hline
		13 & 3238052 & 9321629 & 72 & 4 \\ \hline
		14 & 7199908 & 20573343 & 144 & 4 \\ \hline
	\end{tabular}
	\caption{Classical AC moves}
	\label{fig:classical_persistence}
\end{figure}

\begin{figure}
	\begin{tabular}{|c|c|c|c|c|}
		\hline
		$\ell$ & $v(\ell)$ & $e(\ell)$ & $ic(\ell,1)$ & $ic(\ell,2)$ \\ \hline
		3 & 36 & 40 & 3 & 0 \\ \hline
		4 & 100 & 152 & 3 & 0 \\ \hline
		5 & 388 & 712 & 3 & 0 \\ \hline
		6 & 884 & 1528 & 3 & 0 \\ \hline
		7 & 3892 & 6984 & 3 & 0 \\ \hline
		8 & 9172 & 16728 & 3 & 0 \\ \hline
		9 & 37428 & 69752 & 3 & 0 \\ \hline
		10 & 84996 & 155752 & 3 & 0 \\ \hline
		11 & 350356 & 655928 & 19 & 0 \\ \hline
		12 & 791140 & 1467080 & 67 & 0 \\ \hline
		13 & 3238052 & 6107112 & 243 & 16 \\ \hline
		14 & 7199908 & 13414744 & 483 & 16 \\ \hline
	\end{tabular}
	\caption{Prime AC moves}
	\label{fig:prime_persistence}
\end{figure}

%\[
%%OLD
%\begin{tabular}{|c|c|c|c|c|c|}
%	\hline
%	$\ell$ & $v(\ell)$ & $e(\ell)$ & $ic(\ell,1)$ & $ic(\ell,2)$ & $ic(\ell,3)$ \\ \hline
%	2 & 4 & 4 & 0 & 0 & 0 \\ \hline
%	3 & 36 & 76 & 0 & 0 & 0 \\ \hline
%	4 & 100 & 241 & 0 & 0 & 0 \\ \hline
%	5 & 388 & 1027 & 0 & 0 & 0 \\ \hline
%	6 & 876 & 2224 & 0 & 0 & 0 \\ \hline
%	7 & 3844 & 10057 & 0 & 0 & 0 \\ \hline
%	8 & 8992 & 23726 & 0 & 0 & 0 \\ \hline
%	9 & 35844 & 97243 & 0 & 0 & 0 \\ \hline
%	10 & 81004 & 216412 & 1 & 0 & 0 \\ \hline
%	11 & 338020 & 920347 & 6 & 3 & 0 \\ \hline
%	12 & 762116 & 2043028 & 12 & 8 & 1 \\ \hline
%	13 & 3115928 & 8478633 & 32 & 21 & 2 \\ \hline
%\end{tabular}
%\]

\subsection{Methods}

The number of isolated components with isolation value $k$ in a connected filtered based graph $\Gamma$ is one less than the number of bars in the 0-barcode of its persistent homology.
Since by construction our approximations $\Gamma(\ell)$ are all connected, it suffices to compute said invariant.
For this task we used \texttt{giotto-TDA} version 5.1 \cite{tauzin2021giotto}.
Specifically, its binding of the \texttt{SimplexTree} data structure introduced in \cite{boissonnat2014simplex} and implemented in \texttt{GUDHI} \cite{maria2014gudhi}.
The construction of $\Gamma(\ell)$ is described described in more detail in \cref{s:neighborhoods}. 