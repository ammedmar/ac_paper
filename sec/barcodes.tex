% !TEX root = ../ac_paper.tex

\section{Isolated components and neighborhood sizes}\label{sec:isolated}

\subsection{Isolated components}

In \autoref{ss:greedy_search}, we explored a greedy approach to finding AC-trivializations of a presentation \(\pi\). Specifically, the goal is to construct a sequence of presentations \((\pi_0, \dots, \pi)\), where \(\pi_0\) is the trivial presentation, such that each consecutive pair in the sequence is related by an AC move. Furthermore, at each step \(k\), the presentation \(\pi_k\) is chosen to have the shortest possible length among all presentations connected to \(\pi_{k+1}\) via an AC move.
In general, the length of presentation in an AC-trivialization will increase past the length of presentation being trivialized.
The minimum such increase is a invariant of the presentation which we now study using graph theory ideas from persistence homology.

%\subsubsection{The AC graph}

\subsubsection{Formalization}

A \textit{based graph} is a pair $(\Gamma, v_0)$ consisting of a graph $\Gamma$ and a preferred vertex $v_0$ in it.
We will often drop $v_0$ from the notation.
A \textit{based subgraph} $\Gamma_n$ of $\Gamma$, written $\Gamma_n \leq \Gamma$, is a subgraph $\Gamma_n$ of $\Gamma$ with the same preferred vertex.
We say that $\Gamma_n$ is \textit{full} in $\Gamma$ if for any two vertices in $\Gamma_n$ joined by an edge in $\Gamma$, the edge is also in $\Gamma_n$.
A \textit{filtration} of a based graph $\Gamma$ is a collection
\[
\Gamma_0 \leq \Gamma_1 \leq \Gamma_2 \leq \dotsb
\]
of based subgraphs of $\Gamma$ for which each vertex and edge of $\Gamma$ is in $\Gamma_n$ for some $n$.
We refer to $\Gamma$ as a \textit{filtered based graph}.
If each $\Gamma_n$ is full in $\Gamma$ we refer to the filtration as full and notice that full filtrations are equivalent to $\N$ valued functions from the set of vertices of $\Gamma$ sending $v_0$ to $0$.

Let $\Gamma^{\text{AC(k)}}$ be the graph whose vertices are $k$-balanced presentations, based at the trivial presentation, having an edge between two vertices if there is an AC move between them.
Additionally, $\Gamma^{\text{AC(k)}}$ is equipped with a full filtration obtained from the function sending a vertex to the length of its presentation minus $k$.

%\subsubsection{Isolation values and isolated components}

Given a filtered based graph $(\Gamma, v_0)$.
The \textit{filtration value} $\Filt(v)$ of a vertex $v$ is the smallest $n \in \N$ such that $v$ is a vertex in $\Gamma_n$.
Similarly, its \textit{connectivity value} $\Conn(v)$ is the smallest $n \in \N$ such that $v$ and $v_0$ can be joined by a path in $\Gamma_n$ or is set to $\infty$ if such path does not exist in $\Gamma$.

The \textit{isolation value} of a vertex $v$ in a filtered based graph is defined as
\[
\Isol(v) = \Conn(v) - \Filt(v),
\]
a number in $\N \cup \set{\infty}$.
A vertex is said to be \textit{isolated} if its isolation value is positive.

We introduce an equivalence relation on isolated vertices.
Two belong to the same \textit{isolated component} if they have the same connectivity value, say $n$, and they can be joined by a path in $\Gamma_{n-1}$.
The \textit{isolation value} of a component is the maximum of the isolation values of its elements.

We can interpret these invariants using the framework of topological data analysis.
Specifically, the isolated components of the based filtered graph $\Gamma$ correspond to bars in the barcode of its persistent $0$-homology, with the isolation values of these components corresponding to the lengths of the associated bars.

%Let us recall that in the AC conjecture, the focus is on $k$-balanced presentations and their AC simplifications.
%An AC-simplification of a presentation $\pi$ is a sequence $(\pi_0, \dots, \pi)$ of presentations, with $\pi_0$ being the trivial presentation, such that all consecutive pairs in the sequence are related by an AC move.
%Since $\pi_0$ minimizes length among all balanced presentations, one might adopt a greedy strategy when attempting to simplify a given $\pi$: given $\pi_k$, select $\pi_{k-1}$ such that the length of $\pi_{k-1}$ is either reduced or, at the very least, not increased relative to $\pi_k$.
%
%While this approach proves useful in numerous instances, it is not universally applicable.
%A more nuanced perspective involves considering simplifications $(\pi_0, \dots, \pi_m)$ where the length of each $\pi_k$ does not exceed that of $\pi$.
%Our focus will be on this scenario, referring to presentations that resist such simplifications as \textit{isolated}.
%To quantify the degree of isolation, we introduce an \textit{isolation value} for each presentation, an integer that in extreme cases is $0$ for presentations that are not isolated and $\infty$ for counterexamples to the AC conjecture, if any exist.
%Since isolated presentations are not necessarily isolated from each other, we introduce an equivalence relation that allows us to better count these isolated components and their isolation values.
%
%Employing methodologies from topological data analysis, our objective in this section is to empirically count isolated components and compute their isolation values in increasing neighborhoods of the identity presentation.
%
%\subsection{Formalization}
%
%We provide the basic definitions and conceptual basis for the computations presented in the next subsection.
%
%\subsubsection{Based graphs}
%
%A \textit{based graph} is a pair $(\Gamma, v_0)$ consisting of a graph $\Gamma$ and a preferred vertex $v_0$ in it.
%We will often drop $v_0$ from the notation.
%A \textit{based subgraph} $\Gamma_n$ of $\Gamma$, written $\Gamma_n \leq \Gamma$, is a subgraph $\Gamma_n$ of $\Gamma$ with the same preferred vertex.
%We say that $\Gamma_n$ is \textit{full} in $\Gamma$ if for any two vertices in $\Gamma_n$ joined by an edge in $\Gamma$, the edge is also in $\Gamma_n$.
%A \textit{filtration} of a based graph $\Gamma$ is a collection
%\[
%\Gamma_0 \leq \Gamma_1 \leq \Gamma_2 \leq \dotsb
%\]
%of based subgraphs of $\Gamma$ for which each vertex and edge of $\Gamma$ is in $\Gamma_n$ for some $n$.
%We refer to $\Gamma$ as a \textit{filtered based graph}.
%If each $\Gamma_n$ is full in $\Gamma$ we refer to the filtration as full and notice that full filtrations are equivalent to $\N$ valued functions from the set of vertices of $\Gamma$ sending $v_0$ to $0$.
%
%
%\subsubsection{The AC graph}
%
%Let $\Gamma^{\text{AC(k)}}$ be the graph whose vertices are $k$-balanced presentations, based at the trivial presentation, having an edge between two vertices if there is an AC move between them.
%Additionally, $\Gamma^{\text{AC(k)}}$ is equipped with a full filtration obtained from the function sending a vertex to the length of its presentation minus $k$.

%\subsubsection{Filtration and connectivity values}
%
%Given a filtered based graph $(\Gamma, v_0)$.
%The \textit{filtration value} $\Filt(v)$ of a vertex $v$ is the smallest $n \in \N$ such that $v$ is a vertex in $\Gamma_n$.
%Similarly, its \textit{connectivity value} $\Conn(v)$ is the smallest $n \in \N$ such that $v$ and $v_0$ can be joined by a path in $\Gamma_n$ or is set to $\infty$ if such path does not exist in $\Gamma$.
%
%\subsubsection{Isolation value and isolation components}
%
%The \textit{isolation value} of a vertex $v$ in a filtered based graph is defined as
%\[
%\Isol(v) = \Conn(v) - \Filt(v),
%\]
%a number in $\N \cup \set{\infty}$.
%A vertex is said to be \textit{isolated} if its isolation value is positive.
%
%We introduce an equivalence relation on isolated vertices.
%Two belong to the same \textit{isolated component} if they have the same connectivity value, say $n$, and they can be joined by a path in $\Gamma_{n-1}$.
%The \textit{isolation value} of a component is the maximum of the isolation values of its elements.

\subsubsection{Experimental results}

Let $\Gamma^\ell$ be the full subgraph of $\Gamma^{\text{AC(2)}}$, with the induced filtration, consisting of all presentations with connectivity value less than or equal to $\ell$, i.e., all vertices that can be joined to the trivial one via paths that contains only presentations of length at most~$\ell$.
%Naturally, for any $\ell$ and $n$ we have that $\Gamma^\ell_n$ is a full subgraph of $\Gamma^{\ell+1}_n$.
We will denote by $v(\ell)$ and $e(\ell)$ the number of vertices and edges of $\Gamma(\ell)$.
Let us denote by $ic(\ell)_k$ the number of isolated components with isolation value $k$.
\autoref{fig:classical_persistence} summarize our results for the classic AC moves whereas \autoref{fig:prime_persistence} does so for their prime version.\footnote{For this task we used \texttt{giotto-TDA} version 5.1 \cite{tauzin2021giotto}.
	Specifically, its binding of the \texttt{SimplexTree} data structure introduced in \cite{boissonnat2014simplex} and implemented in \texttt{GUDHI} \cite{maria2014gudhi}.}

\begin{figure}
	\begin{tabular}{|c|c|c|c|c|}
		\hline
		$\ell$ & $v(\ell)$ & $e(\ell)$ & $ic(\ell,1)$ & $ic(\ell,2)$ \\ \hline
		3 & 36 & 72 & 0 & 0 \\ \hline
		4 & 100 & 248 & 0 & 0 \\ \hline
		5 & 388 & 1072 & 0 & 0 \\ \hline
		6 & 884 & 2376 & 0 & 0 \\ \hline
		7 & 3892 & 10775 & 0 & 0 \\ \hline
		8 & 9172 & 25675 & 0 & 0 \\ \hline
		9 & 37428 & 106513 & 0 & 0 \\ \hline
		10 & 84996 & 239733 & 0 & 0 \\ \hline
		11 & 350356 & 1002439 & 4 & 0 \\ \hline
		12 & 791140 & 2251375 & 16 & 0 \\ \hline
		13 & 3238052 & 9321629 & 72 & 4 \\ \hline
		14 & 7199908 & 20573343 & 144 & 4 \\ \hline
	\end{tabular}
	\caption{Classical AC moves}
	\label{fig:classical_persistence}
\end{figure}

\begin{figure}
	\begin{tabular}{|c|c|c|c|c|c|}
		\hline
		$\ell$ & $v(\ell)$ & $e(\ell)$ & $ic(\ell,1)$ & $ic(\ell,2)$ & $ic(\ell,3)$ \\ \hline
		3 & 36 & 40 & 3 & 0 & 0 \\ \hline
		4 & 100 & 152 & 3 & 0 & 0 \\ \hline
		5 & 388 & 712 & 3 & 0 & 0 \\ \hline
		6 & 884 & 1528 & 3 & 0 & 0 \\ \hline
		7 & 3892 & 6984 & 3 & 0 & 0 \\ \hline
		8 & 9172 & 16728 & 3 & 0 & 0 \\ \hline
		9 & 37428 & 69752 & 3 & 0 & 0 \\ \hline
		10 & 84996 & 155752 & 3 & 0 & 0 \\ \hline
		11 & 350356 & 655928 & 19 & 0 & 0 \\ \hline
		12 & 791140 & 1467080 & 67 & 0 & 0 \\ \hline
		13 & 3238052 & 6107112 & 243 & 16 & 0 \\ \hline
		14 & 7199908 & 13414744 & 483 & 16 & 0 \\ \hline
        15 & 29243812 & 55306744 & 1819 & 136 & 32 \\ \hline
\end{tabular}
	\caption{Prime AC moves}
	\label{fig:prime_persistence}
\end{figure}

%\[
%%OLD
%\begin{tabular}{|c|c|c|c|c|c|}
%	\hline
%	$\ell$ & $v(\ell)$ & $e(\ell)$ & $ic(\ell,1)$ & $ic(\ell,2)$ & $ic(\ell,3)$ \\ \hline
%	2 & 4 & 4 & 0 & 0 & 0 \\ \hline
%	3 & 36 & 76 & 0 & 0 & 0 \\ \hline
%	4 & 100 & 241 & 0 & 0 & 0 \\ \hline
%	5 & 388 & 1027 & 0 & 0 & 0 \\ \hline
%	6 & 876 & 2224 & 0 & 0 & 0 \\ \hline
%	7 & 3844 & 10057 & 0 & 0 & 0 \\ \hline
%	8 & 8992 & 23726 & 0 & 0 & 0 \\ \hline
%	9 & 35844 & 97243 & 0 & 0 & 0 \\ \hline
%	10 & 81004 & 216412 & 1 & 0 & 0 \\ \hline
%	11 & 338020 & 920347 & 6 & 3 & 0 \\ \hline
%	12 & 762116 & 2043028 & 12 & 8 & 1 \\ \hline
%	13 & 3115928 & 8478633 & 32 & 21 & 2 \\ \hline
%\end{tabular}
%\]

\subsection{Neighborhoods}

Let us return to our data set of 1190 presentations in the Miller–Schupp series for $n \leq 7$ and $\length(w) \leq 7$.
As explained in \autoref{sec:ppo}, it is split into 417 PPO-solved and 773 PPO-unsolved presentation.
Our goal here is to analyze the relationship between these labels and the sizes of their respective AC neighborhoods, where a presentation is considered to be in the $k$-step neighborhood of another if they can be connected by applying at most $k$ AC moves.

\subsubsection{Experimental results}

There are 131 distinct neighborhood sizes in our data.
% --consult \autoref{s:neighborhoods} for a detailed description of their construction-- 
Their basic statistics are
\[
\begin{tabular}{cccccc}
	\toprule
	\textbf{Min} & \textbf{Max} & \textbf{Mean} & \textbf{Median} \\
	\midrule
	72,964 & 89,872 & 89,532 & 89,859 \\
	\bottomrule
\end{tabular}
\]
A more detailed description of the frequency of values is presented in \autoref{fig:prime_combined_pie}.

\begin{figure}
	\includegraphics[scale=.4]{fig/prime_combined_pie_rl_cropped.pdf}
	\caption{Sizes of the 5-step neighborhood of all considered presentations in the Miller–Schupp series. We group neighborhood sizes whose representation is below 2.5\%.}
	\label{fig:prime_combined_pie}
\end{figure}

The largest neighborhood size accounts for nearly a third of all considered presentations.
Yet, it represents only 2.4\% of solved presentations and almost half of the unsolved presentations, 48.3\% to be exact.
Please consult \autoref{fig:prime_pies} for more details.
We contrast to the situation using BFS, where these numbers are 7.1\% and 52.5\% respectively.

Another notable feature visible in \autoref{fig:prime_pies} is that only three neighborhood sizes account for over three-quarters of all unsolved presentations.
Additionally, considering six sizes, this proportion reaches 96.9\%.
In fact, only twelve neighborhood sizes are featured among unsolved presentations, whereas all 131 sizes appear among neighborhoods of solved presentations.
The most numerous neighborhood size for solved presentations is 89,560, accounting for only 17.3\% of these.
Furthermore, 54.2\% of all solved presentations have a neighborhood size that is shared by less than 2.5\% of solved presentations.

\begin{figure}
	\centering
	\includegraphics[scale=.35]{fig/prime_solved_pie_rl_cropped.pdf}
	\ 
	\includegraphics[scale=.35]{fig/prime_unsolved_pie_rl_cropped.pdf}
	\caption{Pie charts for the neighborhood size of PPO-solved and PPO-unsolved presentations. We grouped sizes with representation below 2.5\%.}
	\label{fig:prime_pies}
\end{figure}

%\begin{figure}[h!]
%	\centering
%	\begin{subfigure}[b]{0.45\textwidth}
	%		\centering
	%		\includegraphics[scale=.4]{fig/prime_solved_pie.pdf}
	%	\end{subfigure}
%	\hfill
%	\begin{subfigure}[b]{0.45\textwidth}
	%		\centering
	%		\includegraphics[scale=.4]{fig/prime_unsolved_pie.pdf}
	%	\end{subfigure}
%	\caption{Pie charts for the neighborhood size of solved and unsolved presentations}
%	\label{fig:prime_pies}
%\end{figure}

As we saw, having maximal neighborhood size provides significant information about the PPO-solved/-unsolved label of a presentation.
Additionally, the minimum neighborhood size among unsolved presentations --89,573-- is also quite informative since 54\% of solved presentations have neighborhood sizes less than that.
We can improve this percentage using that neighborhood sizes of unsolved presentations concentrate in three bands.
Please consult \autoref{fig:prime_histogram}.
We have that 64.3\% of the solved presentations lie outside of the three bands $[89575, 89575]$ \& $[89715, 89831]$ \& $[89844, 89872]$ which contain over 99\% of the unsolved presentations.
By replacing the last band with $[89859,89872]$, the union of these three bands now encompasses the neighborhood sizes of over 90\% of unsolved presentations, while its complement includes 77.2\% of those associated with solved presentations.

\begin{figure}
	\centering
	\includegraphics[scale=.34]{fig/prime_histogram_rl.pdf}
	\includegraphics[scale=.34]{fig/prime_histogram_rl2.pdf}
	\caption{Histograms with 6 and 26 bins respectively of the neighborhood sizes of the 417 PPO-solved and 773 PPO-unsolved presentations.}
	\label{fig:prime_histogram}
\end{figure}

\medskip

One might anticipate that enhancing the discriminatory power of $n$-neighborhoods could be achieved by incorporating features beyond their size.
We explored two additional types of features, but surprisingly, they only marginally improved the accuracy of solved/unsolved predictions.
The first type was based on node centrality, while the second focused on spectral features of the neighborhood graphs.
The latter was particularly intriguing, considering the emphasis on Markov processes and the well-known relationship between random walks on graphs and the graph Laplacian.