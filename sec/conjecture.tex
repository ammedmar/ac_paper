% !TEX root = ../ac_paper.tex

\section{Andrews-Curtis conjecture\label{sec:AC}}

Andrews-Curtis conjecture concerns balanced presentations
of the trivial group.
These are presentations for which the number of relators is
equal to the number of generators.
The conjecture states that any balanced presentation of
the trivial group can be transformed into the trivial presentation $$<x_1, \cdots, x_n | x_1, \cdots, x_n >$$ by
a sequence of the following moves \cite{Andrews-Curtis}.

\begin{itemize}
	\item (concatenation) replace $r_i$ by $r_i r_j$ for $i \neq j$
	\item (inversion) replace $r_i$ by $r_i^{-1}$
	\item (conjugation) replace $r_i$ by $g r_i g^{-1}$ where $g$ is a generator or its inverse.
\end{itemize}

These moves are called AC moves.
The conjecture is widely believed to be false.
\footnote{In \cite{decidability}, it is proven that either Andrews-Curtis conjecture is false, or there exists an algorithm that decides whether a balanced presentation is of the trivial group.
For arbitrary presentations (i.e. not necessarily balanced), the decidability problem is known to be unsolvable.
\fixme{Is Baumslag-Boone-Neumann the right reference for this?}}
Previous approaches to studying this conjecture have made extensive use of classical computer algorithms, such as genetic algorithms \cite{genetic}, and breadth-first-search  \cite{bfs-ac}.
The following result of \cite{bfs-ac} still holds true.

\begin{theorem}
	Let $G$ be a group defined by a balanced presentation on two generators, with the sum
	of the relator lengths at most 13.
	Then:
	\begin{enumerate}[label=(\roman*)]
		\item if $G$ has a trivial abelianization, $G$ is trivial or is isomorphic to $L_2(5)$, the unique perfect
		group of order 120;
		\item if $G$ is trivial, its presentation is AC-equivalent to the standard presentation or to the presentation
		$\langle x, y | x^3 = y^4, xyx = yxy \rangle$.
	\end{enumerate}
\end{theorem}

That is, the simplest potential counterexample of the Andrews-Curtis conjecture for two generators
appears at a total relator length of 13.
This example is part of the Akbulut-Kirby series of potential counterexamples \cite{Akbulut-Kirby}:
\bea
\AK(n) \colon \langle x, y | x^n = y^{n+1}, xyx = yxy\rangle
\eea
for $n \geq 2.$ The case of $n=2$ is known to be consistent with the conjecture; the potential counterexample
in the theorem above is the case of $n=3$.
Previous works have verified that no sequence of AC moves that allows the length of each relator to grow up to 20 trivializes this presentation \cite{Panteleev-Ushakov}.
Naively, this could be taken as strong evidence in favor of $\AK(3)$ being a true counterexample of the AC conjecture.
However, it should be interpreted with caution as there exist AC-trivializable presentations of the trivial group for which the number of AC moves grows more quickly than any tower of exponentials \cite{Bridson, Lishak}.
Roughly speaking, for these presentations, if the sum of word lengths is $n$, the number of AC moves required to trivialize the presentation is at least $\Delta (\lfloor \log_2 (n) \rfloor)$ where $\Delta \colon \mathbb{N} \to \mathbb{N}$ is defined recursively as $\Delta(0) = 2$ and $\Delta (m) = 2^{\Delta(m-1)}$ for $m \geq 1$.
\newline

Another well-known series of potential counterexamples is due to Miller and Schupp \cite{Miller-Schupp}:
\bea
\MS(n, w): \ \langle x, y | x^{-1} y^n x = y^{n+1}, x = w \rangle.
\eea
Here $n > 0$ and $w$ is a word in $x$ and $y$ such that $w$ has exponent sum 0
on $x$.
\footnote{Angus has pointed out that for $n=1$, these are all likely trivializable.
Can we show that?
	\fixme{I think we should be able to list all possibilities for w; this seems like a combinatorial problem.} }
\footnote{The first potential counterexamples in this series that are not AC-equivalent to the standard presentation or to $AK(3)$ would be of length 14.
Can we list those presentations and	check their relation to the trivial presentation or AK(3)? One such element is the case $n=3$, $w = y^{-1} x y x^{-1}$; this example is AC-trivializable \cite{morse}.}
For $w_\star = y^{-1} x^{-1} y x y$, $\MS(w_\star, n)$ is known to be AC-equivalent to $\AK(n)$ \cite{MMS}; therefore, in this paper, we focus entirely on the Miller-Schupp series.
Notice that the sum of relator lengths of $\AK(n)$ is $2n+7$, and that of $\MS(w_\star, n)$ is $2n+9$.
\newline

A third series of potential counterexamples is due to Gordon \cite{Brown}:
\bea
\G(m, n, p, q): \langle x, y | x = [x^m, y^n], y = [y^p, x^q] \rangle.
\eea
where $m,n,p,q \in \mathbb{Z}$.
All examples of this series with the total relator length of 14 are known
to be AC-trivializable \cite{Bowman-McCaul}
\footnote{Perform the same exercise as in the footnote above for this series.}
\newline

There is also a length-14 presentation in \cite{MMS} for which the authors could not reduce the length of the relators using AC moves.
\bea
<x, y | xyx^{-2}y^{-1} xy^{-1}, x^{-1} y^{-1} x y^2 x y^{-1}>.
\eea

\subsection{Generalizations and other variants}

There also exists a weaker version of the Andrews-Curtis conjecture, also known as the "stable Andrews-Curtis conjecture".
It allows, in addition to the AC moves presented above, a "stability move" --- i.e., extend a presentation $\langle x_1, \cdots, x_n | r_1, \cdots, r_n \rangle$ to include more generators:
\bea
\langle x_1, \cdots, x_n, x_{n+1}, \cdots x_m | r_1, \cdots, r_n, x_{n+1}, \cdots, x_m \rangle.
\eea

It is not known if stable AC-equivalence between two presentations implies AC-equivalence between the two presentations.
A potential counterexample to this statement is given in \cite{MMS}: the presentation $\langle x, y | x^4 = y x^2 y^{-1} x^{-1} y x^2 y^{-1}, y = [x^2, y]^3 $ of total word length 32 is stably AC-equivalent to $\AK(3)$ but it is not known if the two presentations are AC-equivalent.
\footnote{Exploring this presentation with our RL agent could be useful in showing that the two are indeed equivalent.}
\newline

There also exist other generalizations of the conjecture for other kinds of groups; some of which have been proved.
In particular, the conjecture holds true for finite groups and soluble groups \cite{Borovik, Guyot}.
I think these versions will not be important for us.
\newline

Finally, there exists a notion of "elementary M-transformations" which could be useful \cite{BurnsI, BurnsII}.
Each elementary M-transformation is a composition of AC moves, such that trivializing a presentation requires dramatically less M-transformations compared to the number of AC moves.
For example, trivializing $\AK(2)$ requires 14 AC moves, but only 2 elementary M-transformations.\fixme{Explain M-transformations a bit here.}