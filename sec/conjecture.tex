% !TEX root = ../ac_paper.tex

\section{Andrews-Curtis conjecture\label{sec:AC}}

The Andrews-Curtis conjecture concerns the study of balanced presentations of the trivial group, i.e. presentations of the trivial group with equal number of generators and relators. The conjecture proposes that any balanced presentation 
\[
\angles{x_1, \dots, x_n \mid r_1, \dots, r_n}
\]
can be converted to the trivial presentation
\[
\angles{x_1, \dots, x_n \mid x_1, \dots, x_n}
\]
through a series of following operations known as AC moves \cite{Andrews-Curtis}.
\begin{enumerate}
	\item Substitute some $r_i$ by $r_i r_j$ for $i \neq j$.
	\item Replace some $r_i$ by $r_i^{-1}$.
	\item Change some $r_i$ to $g r_i g^{-1}$ where $g$ is a generator or its inverse.
\end{enumerate}

Little headway has been made in establishing a proof of the conjecture. On the other hand, several families of potential counterexamples have been suggested in the literature; we collect a subset of these in Appendix A. \footnote{\fixme{Include a reference.}} To investigate a given presentation, one can systematically explore the entire space of possible sequences of AC moves in search of a sequence that renders the presentation trivial. However, this space grows exponentially with the length of the sequence. For a presentation with $n$ generators, there are $3n^2$ AC moves, and the total number of sequences of AC moves of length $k$ is $(3n^2)^k$. Even for a modest case like $n=2$ and $k=20$, the number is on the order of $10^{21}$, making a brute-force approach impractical.
\newline 

Cassical search algorithms such as genetic algorithms \cite{genetic}, and breadth-first-search  \cite{bfs-ac} have been employed to address this challenge. These methods have shown success in trivializing balanced presentations with two generators and total lengths less than 13. The following presentation of length 13,
\[
\angles{x, y \mid x^3 = y^4, xyx = yxy}
\]
represents the shortest presentation, upto AC-equivalence, that alludes all attempts at length reduction. This presentation is a part of an infinite series of potential counterexamples by Akbulut and Kirby \cite{Akbulut-Kirby}:
\[
\AK(n) \colon \angles{x, y \mid x^n = y^{n+1}, xyx = yxy}, \quad n \geq 2
\]
$\AK(2)$ has length 11 and has been established as AC-trivial  \cite{genetic}. $\AK(3)$ is the afore-mentioned length-13 presentation. 
\newline

In slightly over two decades since the first utilization of search algorithms \cite{genetic, bfs-ac}, unsuccessful attempts have been made to trivialize $\AK(3)$ with similar search algorithms and with an increased amount of computational resources \cite{Bowman-McCaul, krawiec2016distance, Panteleev-Ushakov}. Notably, \cite{Panteleev-Ushakov} found that no sequence of AC moves that allows relator lengths to increase up to 20 trivializes $\AK(3)$. 
This lack of success could be interpreted as suggestive evidence that $\AK(3)$ might be as a counterexample to the Andrews-Curtis conjecture.
\newline 

However, recent works by Bridson and Lishak have showed that there exist AC-trivializable balanced presentations of the trivial group, for which the number of AC moves in a trivializing sequence is bounded below by a superexponential function of the length of the presentation \cite{Bridson, Lishak}.
Roughly speaking, for these presentations, if the sum of word lengths is $k$, the number of AC moves required to trivialize the presentation is at least $\Delta (\lfloor \log_2 k \rfloor)$ where $\Delta \colon \mathbb{N} \to \mathbb{N}$ is defined recursively as $\Delta(0) = 2$ and $\Delta (j) = 2^{\Delta(j-1)}$ for $j \geq 1$. 
\footnote{For $k=13$, i.e. the length of $\AK(3)$, $\Delta (\lfloor \log_2 (k) \rfloor) = 65536$. In contrast, presentations trivialized by search algorithms have AC sequences of length less than $10^3$. The main reason, as mentioned above, is that the number of sequences of AC moves grows exponentially with the length of the sequence, and even with large amounts of storage in modern computers, we are not able to search through the space efficiently.}
While $\AK(3)$ was not directly examined by Boris and Lishak, their findings challenge the inclination to view it as a counterexample. Furthermore, their work underscores the necessity of employing novel computational methods to deepen our understanding of the AC-conjecture.
\newline

In this paper, we will consider a variety of computational tools to understand better the properties of balanced presentations of the trivial group. We will test the efficacy of our approaches on a subset of presentations from the Miller-Schupp series of potential counterexamples \cite{Miller-Schupp}.
\[
\MS(n, w) \colon \angles{x, y \mid x^{-1} y^n x = y^{n+1}, x = w}.
\]
Here, $n > 0$, and $w$ is a word in $x$ and $y$ with zero exponent sum on $x$. For $w_\star = y^{-1} x^{-1} y x y$, the presentations $\MS(n, w_\star)$ are AC-equivalent to the presentations from Akbulut-Kirby series \cite{MMS}. In particular, the presentation
\[
\MS(n, w) \colon \angles{x, y \mid x^{-1} y^3 x = y^{4}, x =  y^{-1} x^{-1} y x y}.
\]
of length 15 is AC-equivalent to $\AK(3)$. We will only consider presentations where $n \leq 7$ and the length of words $w$ does not exceed 7.
Our selection criteria aimed to strike a balance: we sought a dataset of presentations large enough to allow for meaningful analysis and conclusions regarding our methodologies, yet compact enough to ensure that our computational tasks remain feasible within a practical timeframe.
We reduced $x^{-1}w$ freely and cyclically, and kept only one representative of each orbit under the action of cyclic permutations. After these simplifications, we were left with $170$ choices for $x^{-1} w$. This resulted in a dataset of $7 \times 170 = 1190$ presentations from the Miller-Schupp series.
%\footnote{Notice that the sum of relator lengths of $\AK(n)$ is $2n+7$, and that of $\MS(w_\star, n)$ is $2n+9$.}
%\footnote{We should be able to show that all $n=1$ presentations are trivializable.}
\newline

Our implementation of moves differed slightly from the afore-mentioned AC moves.  We considered the following set of transformations, which generates a group isomorphic to the group generated by AC moves for $n=2$.
\footnote{For a given presentation $\angles{x_1, x_2 \mid r_1, r_2}$, the sequence of following four moves: $r_2 \to r_2 r_1$, $r_1 \to r_1 r_2^{-1}$, $r_2 \to r_2 r_1$, and $r_2 \to r_1 r_2 r_1^{-1}$ results in the presentation $\angles{x_1, x_2 \mid r_2^{-1}, r_1}$. This is equivalent to the AC move $r_2 \to r_2^{-1}$ followed by an operation swapping the two relators. We also enhanced the notion of trivial presentation to mean the following set of eight presentations of length 2: $\{\angles{x_1, x_2 \mid x_1^{\pm 1}, x_2^{\pm 1}} , \angles{x_1, x_2 \mid x_2^{\pm 1}, x_1^{\pm 1}}  \}$.}
\begin{enumerate}
	\item Substitute some $r_i$ by $r_i r_j^{\pm 1}$ for $i \neq j$.
	\item Change some $r_i$ to $g r_i g^{-1}$ where $g$ is a generator or its inverse.
\end{enumerate}

We found this particular choice of transformations to be more useful in early experiments with reinforcement-learning. 
\footnote{In reinforcement learning, we need a reward function to search through the set of all possible AC moves. A natural reward function is the negative of the total length of a presentation. An inversion move leaves the length of a presentation invariant providing no signal to the training process, and hence significantly affects the performance of an agent. This issue is discussed in more detail in section 5.}
From here onwards in this paper, we will refer to these transformations as ``AC moves" instead of the original AC moves mentioned above.
%\footnote{\fixme{We should run some experiments that show how the use of these moves as compared to the original AC moves affects the path lengths in BFS.}}