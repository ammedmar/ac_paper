% !TEX root = ../ac_paper.tex

\section{Andrews-Curtis conjecture\label{sec:AC}}

The Andrews-Curtis conjecture concerns the study of balanced presentations of the trivial group, i.e. presentations of the trivial group with equal number of generators and relators. The conjecture proposes that any balanced presentation 
\[
\angles{x_1, \dots, x_n \mid r_1, \dots, r_n}
\]
can be converted to the trivial presentation
\[
\angles{x_1, \dots, x_n \mid x_1, \dots, x_n}
\]
through a series of following operations known as AC moves \cite{Andrews-Curtis}.

\begin{enumerate}
	\item Substitute some $r_i$ by $r_i r_j$ for $i \neq j$.
	\item Replace some $r_i$ by $r_i^{-1}$.
	\item Change some $r_i$ to $g r_i g^{-1}$ where $g$ is a generator or its inverse.
	\item Reorder the relators.
\end{enumerate}

Previous approaches to studying this conjecture have made extensive use of classical computer algorithms, such as genetic algorithms \cite{genetic}, and breadth-first-search  \cite{bfs-ac}. These approaches successfully trivialized balanced presentations with two generators whose sum of lengths of relators is at most 12. The following presentation of length 13,
\[
\angles{x, y \mid x^3 = y^4, xyx = yxy}
\]
was discovered to be the shortest presentation (upto AC-equivalence) that alluded all attempts of reducing length. This presentation was previously suggested as a member of an inifite series of potential counterexamples by Akbulut and Kirby \cite{Akbulut-Kirby}:

\[
\AK(n) \colon \angles{x, y \mid x^n = y^{n+1}, xyx = yxy} \quad , \quad n \geq 2
\]
The case of $n=2$ is a length-11 presentation and is now known to be AC-trivial. The length-13 presentation mentioned above is the case $n=3$. 
\newline

In the two decades since \cite{genetic, bfs-ac}, breadth-first search has been applied to trivialize this length-13 presentation with larger amounts of compute and storage but has had no signs of success \cite{Panteleev-Ushakov}. 
\footnote{Include citations to other papers.}
In particular, the work of \cite{Panteleev-Ushakov} showed that no sequence of AC moves that allows relator lengths to increase up to 20 trivializes this presentation. This could be taken as  evidence that the length-13 presentation is, in fact, a counterexample to Andrews-Curtis conjecture. However, recent works of Bridson and Lishak have showed that there exist AC-trivializable presentations of the trivial group for which the number of AC moves in a trivializing sequence is bounded below by a superexponential function of the length of the presentation \cite{Bridson, Lishak}.
Roughly speaking, for these presentations, if the sum of word lengths is $n$, the number of AC moves required to trivialize the presentation is at least $\Delta (\lfloor \log_2 (n) \rfloor)$ where $\Delta \colon \mathbb{N} \to \mathbb{N}$ is defined recursively as $\Delta(0) = 2$ and $\Delta (m) = 2^{\Delta(m-1)}$ for $m \geq 1$.
\newline

In this paper, we will consider a wide range of computational tools to understand the properties of balanced presentations of trivial group. In order to 

consider a wider class of potential counterexamples, known as the Miller-Schupp series \cite{Miller-Schupp}. It consists of presentations of the form
\[
\MS(n, w) \colon \angles{x, y \mid x^{-1} y^n x = y^{n+1}, x = w}.
\]
where $n > 0$, and $w$ is a word in $x$ and $y$ such that with zero exponent sum on $x$.
\footnote{Angus has pointed out that for $n=1$, these are all likely trivializable.
Can we show that? 
I think we should be able to list all possibilities for w; this seems like a combinatorial problem.
Also, I think that it is worthwhile to compare RL with BFS just for these examples. If RL solves these examples in lesser number of moves, perhaps it can solve more of them. 
This is because BFS trees will get too large as the length of $w$ is increased but RL does not have this problem.
}
\footnote{The first potential counterexamples in this series that are not AC-equivalent to the standard presentation or to $AK(3)$ would be of length 14.
Can we list those presentations and	check their relation to the trivial presentation or AK(3)? One such element is the case $n=3$, $w = y^{-1} x y x^{-1}$; this example is AC-trivializable \cite{morse}.}
The presentations $\MS(n, w_\star)$ for $w_\star = y^{-1} x^{-1} y x y$ are known to be AC-equivalent to $\AK(n)$. In particular, $\AK(3)$ studied above is  \cite{MMS}; therefore, in this paper, we focus entirely on the Miller-Schupp series.
Notice that the sum of relator lengths of $\AK(n)$ is $2n+7$, and that of $\MS(w_\star, n)$ is $2n+9$.


We consider examples of the Miller-Schupp series for $n \leq 7$ and $\text{len}(w) \leq 7$. \fixme{How many examples are there?}
 
 \fixme{Give details of the specific way in which we coded AC moves.}

We consider all examples of the Miller-Schupp series for $n \leq 7$ and $\text{len}(w) \leq 7$.
We freely and cyclically reduce $x^{-1}w$ and, keep only one element of each orbit under the action of cyclic permutations.
\footnote{Here, the restriction len$(w) \leq 7$ was imposed \textit{before} freely and cyclically reducing $x^{-1}w$.
The total length of a presentation in the Miller-Schupp series before this reduction is $2n+4+\text{len}(w)$, however, after these reductions, the total length decreases to $2n+4+\text{len}(w')$ where $x^{-1}w'$ is the reduced word.
This can be a source of confusion.
Maybe I should consider all examples that satisfy $\text{len}(w') \leq 7$.}
There are 170 relators of the form $x^{-1}w$ with these restrictions, and hence 1190 presentations in total.

%\footnote{In \cite{decidability}, it is proven that either Andrews-Curtis conjecture is false, or there exists an algorithm that decides whether a balanced presentation is of the trivial group.
%For arbitrary presentations (i.e. not necessarily balanced), the decidability problem is known to be unsolvable.
%fixme{Is Baumslag-Boone-Neumann the right reference for this?}}