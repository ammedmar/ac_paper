% !TEX root = ../ac_paper.tex

\section{Andrews-Curtis conjecture\label{sec:AC}}

The Andrews-Curtis conjecture concerns the study of balanced presentations of the trivial group, i.e. presentations of the trivial group with equal number of generators and relators. The conjecture proposes that any balanced presentation 
\[
\angles{x_1, \dots, x_n \mid r_1, \dots, r_n}
\]
can be converted to the trivial presentation
\[
\angles{x_1, \dots, x_n \mid x_1, \dots, x_n}
\]
through a series of following operations known as AC moves \cite{Andrews-Curtis}.

\begin{enumerate}
	\item Substitute some $r_i$ by $r_i r_j$ for $i \neq j$.
	\item Replace some $r_i$ by $r_i^{-1}$.
	\item Change some $r_i$ to $g r_i g^{-1}$ where $g$ is a generator or its inverse.
	\item Reorder the relators.
\end{enumerate}

Previous approaches to studying this conjecture have made extensive use of classical computer algorithms, such as genetic algorithms \cite{genetic}, and breadth-first-search  \cite{bfs-ac}. These approaches successfully trivialized balanced presentations with two generators whose sum of lengths of relators is at most 12. The following presentation of length 13,
\[
\angles{x, y \mid x^3 = y^4, xyx = yxy}
\]
was discovered to be the shortest presentation (upto AC-equivalence) that alluded all attempts of reducing length. This presentation was previously suggested as a member of an inifite series of potential counterexamples by Akbulut and Kirby \cite{Akbulut-Kirby}:
\[
\angles{x, y \mid x^n = y^{n+1}, xyx = yxy}, \quad n \geq 2
\]
For $n=2$, it gives a length-11 presentation, which was later shown to be AC-trivial \cite{genetic}. The case $n=3$ is the length-13 presentation mentioned above. 
\newline

In a little more than two decades since \cite{genetic, bfs-ac}, search algorithms have been used with larger amounts of compute and storage in attempts to trivialize this length-13 presentation but have had no success \cite{Bowman-McCaul, krawiec2016distance, Panteleev-Ushakov}. 
In particular, the work of \cite{Panteleev-Ushakov} showed that no sequence of AC moves that allows relator lengths to increase up to 20 trivializes this presentation. This could be taken as  evidence that the length-13 presentation is, in fact, a counterexample to Andrews-Curtis conjecture. However, recent works of Bridson and Lishak have showed that there exist AC-trivializable balanced presentations of the trivial group for which the number of AC moves is bounded below by a superexponential function of the length of the presentation \cite{Bridson, Lishak}.
Roughly speaking, for these presentations, if the sum of word lengths is $n$, the number of AC moves required to trivialize the presentation is at least $\Delta (\lfloor \log_2 (n) \rfloor)$ where $\Delta \colon \mathbb{N} \to \mathbb{N}$ is defined recursively as $\Delta(0) = 2$ and $\Delta (m) = 2^{\Delta(m-1)}$ for $m \geq 1$.
%\footnote{Can we write concisely the family of presentations studied by Lishak? How different are they from AK(n)? }
\footnote{Note that for $n=13$, $\Delta (\lfloor \log_2 (n) \rfloor) = 65536$.  \fixme{This could actually be used in the introduction where we argue for the potential usefulness of finding supermoves.}}
\newline

In this paper, we will consider a wide range of computational tools to understand better the properties of balanced presentations of the trivial group. We will test the efficacy of our approaches on a subset of presentations from the Miller-Schupp series of potential counterexamples \cite{Miller-Schupp}.
\[
\MS(n, w) \colon \angles{x, y \mid x^{-1} y^n x = y^{n+1}, x = w}.
\]
Here, $n > 0$, and $w$ is a word in $x$ and $y$ with zero exponent sum on $x$. For $w_\star = y^{-1} x^{-1} y x y$, the presentations $\MS(n, w_\star)$ are AC-equivalent to the presentations from Akbulut-Kirby series \cite{MMS}. In particular, the presentation
\[
\MS(n, w) \colon \angles{x, y \mid x^{-1} y^3 x = y^{4}, x =  y^{-1} x^{-1} y x y}.
\]
of length 15 is AC-equivalent to the shortest potential counterexample mentioned above. We will restrict to the cases $n \leq 7$ and also only consider words $w$ of length at most 7. 
\footnote{These choices are somewhat arbitrary. We wanted a set of presentations that is large enough for us to be able to make reasonable conclusions about our approaches, but still small enough that we can perform computations in a reasonable amount of time.  \fixme{Why not restrict length to 25 instead of restricting n and length of w? Well if we do that with 2n + 4 + lenw = 25, then lenw has to be at least 1, so n can be at most 10. and n has to be at least 1, so lenw can be at most 19.}}
We reduced $x^{-1}w$ freely and cyclically, and kept only one representative of each orbit under the action of cyclic permutations. After these simplifications, we were left with $170$ choices for $x^{-1} w$. In overall, this amounted to $7 \times 170 = 1190$ presentations from the Miller-Schupp series.
%\footnote{Notice that the sum of relator lengths of $\AK(n)$ is $2n+7$, and that of $\MS(w_\star, n)$ is $2n+9$.}
\footnote{Angus has pointed out that for $n=1$, these are all likely trivializable.
Can we show that? 
I think we should be able to list all possibilities for w; this seems like a combinatorial problem.
Also, I think that it is worthwhile to compare RL with BFS just for these examples. If RL solves these examples in lesser number of moves, perhaps it can solve more of them. 
This is because BFS trees will get too large as the length of $w$ is increased but RL does not have this problem.
}
\footnote{The first potential counterexamples in this series that are not AC-equivalent to the standard presentation or to $AK(3)$ would be of length 14.
Can we list those presentations and	check their relation to the trivial presentation or AK(3)? One such element is the case $n=3$, $w = y^{-1} x y x^{-1}$; this example is AC-trivializable \cite{morse}.}
\newline




\fixme{This should go somewhere. For a balanced presentation of rank n, there are 3n2 AC-moves, and hence (3n2)l move sequences of length l. Many move sequences will result in the same presentation so the value of (3n2)l is really only a rough upper bound.} 


 
 \fixme{Give details of the specific way in which we coded AC moves.}
 
 \fixme{How long did a presentation have to get before it could be trivialized? Perhaps this question should be moved to the section on BFS. It should also be answered with RL. A plot that shows this result will be useful.}


%\footnote{In \cite{decidability}, it is proven that either Andrews-Curtis conjecture is false, or there exists an algorithm that decides whether a balanced presentation is of the trivial group.
%For arbitrary presentations (i.e. not necessarily balanced), the decidability problem is known to be unsolvable.
%fixme{Is Baumslag-Boone-Neumann the right reference for this?}}