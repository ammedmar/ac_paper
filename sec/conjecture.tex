% !TEX root = ../ac_paper.tex

\section{Andrews--Curtis conjecture\label{sec:AC}}

The Andrews--Curtis conjecture concerns the study of \textit{balanced presentations} of the trivial group, i.e.
presentations of the trivial group with an equal number of generators and relators.
The conjecture proposes that any balanced presentation
\[
\angles{x_1, \dots, x_n \mid r_1, \dots, r_n}
\]
can be converted to the trivial presentation
\[
\angles{x_1, \dots, x_n \mid x_1, \dots, x_n}
\]
through a series of the following operations known as \textit{AC moves} \cite{Andrews--Curtis}.
\begin{enumerate}[label=(AC\arabic*)]
	\item Substitute some $r_i$ by $r_i r_j$ for $i \neq j$.
	\item Replace some $r_i$ by $r_i^{-1}$.
	\item Change some $r_i$ to $g r_i g^{-1}$ where $g$ is a generator or its inverse.
\end{enumerate}
We will refer to the sum of the word lengths of all relators as the \textit{length} of a presentation.
Two presentations that can be transformed into each other by a sequence of AC moves are said to be \textit{AC-equivalent}.
A presentation that is AC-equivalent to the trivial presentation is referred to as \textit{AC-trivial}.
Despite considerable efforts, little progress has been made in establishing a proof of the conjecture.
However, several families of potential counterexamples have been suggested in the literature.

To investigate a given presentation, one may systematically explore the entire space of possible sequences of AC moves in search of a sequence that renders the presentation trivial.
This space grows exponentially with the length of the sequence.
For a presentation with $n$ generators, there are $3n^2$ AC moves, and the total number of sequences of AC moves of length $k$ is $(3n^2)^k$.
Even for a modest case like $n=2$ and $k=20$, the number of possible sequences is on the order of $10^{21}$, making a brute-force approach impractical.


Classical search algorithms such as genetic algorithms \cite{genetic}, and breadth-first search \cite{bfs-ac} have been employed to search through this space and achieved success in trivializing balanced presentations with two generators and lengths less than 13.
The following presentation of length 13,
\[
\angles{x, y \mid x^3 = y^4, xyx = yxy}
\]
is the shortest presentation, up to AC-equivalence, that eludes all attempts at length reduction.
This presentation is a part of an infinite series of potential counterexamples by Akbulut and Kirby \cite{Akbulut--Kirby}:
\[
\AK(n) = \angles{x, y \mid x^n = y^{n+1}, xyx = yxy}, \quad n \geq 2.
\]
$\AK(2)$ has length 11 and has been established as AC-trivial \cite{genetic} whereas $\AK(3)$ is the aforementioned presentation with length 13.


In over two decades since the first utilization of search algorithms \cite{genetic, bfs-ac}, only unsuccessful attempts have been made to trivialize $\AK(3)$ with different variants of breadth-first search algorithm using an increased amount of computational resources \cite{Bowman-McCaul, krawiec2016distance, Panteleev-Ushakov}.
Notably, \cite{Panteleev-Ushakov} found that no sequence of AC moves that allows relator lengths to increase up to 20 trivializes $\AK(3)$.
This lack of success could be interpreted as suggestive evidence that $\AK(3)$ might be a counterexample to the Andrews--Curtis conjecture.
However, recent works by Bridson and Lishak have shown that there exist AC-trivializable balanced presentations of the trivial group, for which the number of AC moves in a trivializing sequence is bounded below by a superexponential function of the length of the presentation \cite{Bridson, Lishak}.
Roughly speaking, for these presentations, if the sum of word lengths is $k$, the number of AC moves required to trivialize the presentation is at least $\Delta (\lfloor \log_2 k \rfloor)$ where $\Delta \colon \mathbb{N} \to \mathbb{N}$ is defined recursively as $\Delta(0) = 2$ and $\Delta (j) = 2^{\Delta(j-1)}$ for $j \geq 1$.
In particular, $\Delta (\lfloor \log_2 (13) \rfloor) = 65536$, whereas presentations trivialized by the aforementioned search algorithms have AC sequences of length less than $1000$.
While $\AK(3)$ is itself not a member of the family of examples studied by Boris and Lishak, their findings challenge the inclination to view it as a counterexample.
Their work also underscores the necessity of employing search methods that are more efficient than breadth-first search.


In this paper, we will consider a variety of computational tools to better understand the properties of balanced presentations of the trivial group.
We will test the efficacy of our approaches on a subset of presentations from the Miller--Schupp series of potential counterexamples \cite{Miller--Schupp}:
\[
\MS(n, w) = \angles{x, y \mid x^{-1} y^n x = y^{n+1}, x = w}.
\]
Here, $n > 0$, and $w$ is a word in $x$ and $y$ with zero exponent sum on $x$.
For $w_\star = y^{-1} x^{-1} y x y$, the presentations $\MS(n, w_\star)$ are AC-equivalent to the presentations from Akbulut--Kirby series \cite{MMS}.
In particular, the presentation
\[
\MS(n, w) = \angles{x, y \mid x^{-1} y^3 x = y^{4}, x =  y^{-1} x^{-1} y x y}.
\]
of length 15 is AC-equivalent to $\AK(3)$.

We will only consider presentations with $n,\, \length(w) \leq 7$.
Our selection criteria aimed to strike a balance: we sought a dataset of presentations large enough to allow for meaningful analysis, yet small enough to ensure all computations are feasible within a practical timeframe.
Additionally, we reduced $x^{-1}w$ freely and cyclically and kept only one representative of each orbit under the action of cyclic permutations.\anibal{What is the action?}
After these simplifications, we were left with $170$ choices for $x^{-1} w$.
This resulted in a dataset of $7 \times 170 = 1190$ presentations from the Miller--Schupp series.


Our implementation of AC transformations differed from the AC transformations mentioned above in two ways.
First, we considered the following set of operations.
\begin{enumerate}[label=(AC$'$\arabic*)]
	\item Substitute some $r_i$ by $r_i r_j^{\pm 1}$ for $i \neq j$.
	\item Change some $r_i$ to $g r_i g^{-1}$ where $g$ is a generator or its inverse.
\end{enumerate}
For two generators, which is the only case we study in this paper, the group generated by these AC transformations is isomorphic to the group generated by the original AC transformations.\footnote{
The difference lies in how the inversion of a relator is handled: we always follow an inversion by a concatenation, while the original AC moves allow for standalone inversion moves.
The original inversion moves may be retrieved from the new generators as follows.
For a given presentation $\angles{x_1, x_2 \mid r_1, r_2}$, the sequence of moves: $r_2 \to r_2 r_1$, $r_1 \to r_1 r_2^{-1}$, $r_2 \to r_2 r_1$, and $r_2 \to r_1 r_2 r_1^{-1}$ results in the presentation $\angles{x_1, x_2 \mid r_2^{-1}, r_1}$, which is the same as $r_2 \to r_2^{-1}$ up to swapping the two relators.
We also enhanced the notion of trivial presentation(s) to include all presentations of length 2: $\{\angles{x_1, x_2 \mid x_i^{a}, x_j^{b}}  \mid i, j = 1, 2; a, b = \pm 1; i \neq j \}$.
}
The reason for this change is due to its effect on performance in greedy search and reinforcement learning algorithms studied in \autoref{sec:search} and \autoref{sec:rl}.
Specifically, the length of a presentation provides a useful signal when searching through the space of presentations with these algorithms.
An inversion transformation leaves the length invariant providing no signal to the search process and slowing down the performance of the algorithm significantly.
For the rest of the paper we will refer to the new transformations (instead of the original AC transformations) as ``AC transformations" or ``AC moves".

Second, in order to make the search space finite in size, we set a maximum length that each relator is allowed to take.
If an AC transformation resulted in a presentation with a relator of length greater than this maximum length, the AC transformation was set to act trivially.
In the search of a sequence of AC moves that trivialize a presentations of the Miller--Schupp series $\MS(n, w)$, we set this maximum length to be $2 \times \text{max}(2 n+3, \length(w)+1) + 2$.
This specific choice was made to allow for at least one concatenation move followed by a conjugation move in the search process.
%Note that this constraint is quite restrictive: any presentation of the Miller--Schupp series that requires the length of a relator to grow to more than the set maximum length would not be trivialized.