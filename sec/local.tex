% !TEX root = ../ac_paper.tex

\subsection{Neighborhoods}

In this subsection we consider the size of neighborhoods of presentations.
A presentation is considered to be in the $k$-step neighborhood of another if they can be connected by applying at most $k$ AC moves.\footnote{For the experiments below we use the ``prime" AC moves as defined in \cref{?}. The results obtained using the classic AC moves are similar.}

In the previous subsections we explained how, using Proximal Policy Optimization, we labeled each of the presentations in the Miller–Schupp series for $n \leq 7$ and $\length(w) \leq 7$ as either ``solved" or ``unsolved," resulting in a total of 417 and 773 presentations respectively.
Our goal here is to analyze the relationship between these labels and the sizes of their respective 5-step neighborhoods.

There are 131 distinct neighborhood sizes in our data --consult \cref{s:neighborhoods} for a detailed description of their construction-- their basic statistics are
\[
\begin{tabular}{cccccc}
	\toprule
	\textbf{Min} & \textbf{Max} & \textbf{Mean} & \textbf{Median} \\
	\midrule
	72,964 & 89,872 & 89,532 & 89,859 \\
	\bottomrule
\end{tabular}
\]
A more detailed description of the frequency of values is presented in \cref{fig:prime_combined_pie}.

\begin{figure}
	\includegraphics[scale=.4]{fig/prime_combined_pie_rl_cropped.pdf}
	\caption{Sizes of the 5-step neighborhood of all considered presentations in the Miller–Schupp series. We group neighborhood sizes whose representation is below 2.5\%.}
	\label{fig:prime_combined_pie}
\end{figure}

The largest neighborhood size accounts for nearly a third of all considered presentations.
Yet, it represents only 2.4\% of solved presentations and almost half of the unsolved presentations, 48.3\% to be exact.
Please consult \cref{fig:prime_pies} for more details.
We contrast to the situation using BFS, where these numbers are 7.1\% and 52.5\% respectively.

Another notable feature visible in \cref{fig:prime_pies} is that only three neighborhood sizes account for over three-quarters of all unsolved presentations.
Additionally, considering six sizes, this proportion reaches 96.9\%.
In fact, only twelve neighborhood sizes are featured among unsolved presentations, whereas all 131 sizes appear among neighborhoods of solved presentations.
The most numerous neighborhood size for solved presentations is 89,560, accounting for only 17.3\% of these.
Furthermore, 54.2\% of all solved presentations have a neighborhood size that is shared by less than 2.5\% of solved presentations.

\begin{figure}
	\centering
	\includegraphics[scale=.35]{fig/prime_solved_pie_rl_cropped.pdf}
	\ 
	\includegraphics[scale=.35]{fig/prime_unsolved_pie_rl_cropped.pdf}
	\caption{Pie charts for the neighborhood size of PPO-solved and PPO-unsolved presentations. We grouped sizes with representation below 2.5\%.}
	\label{fig:prime_pies}
\end{figure}

%\begin{figure}[h!]
%	\centering
%	\begin{subfigure}[b]{0.45\textwidth}
%		\centering
%		\includegraphics[scale=.4]{fig/prime_solved_pie.pdf}
%	\end{subfigure}
%	\hfill
%	\begin{subfigure}[b]{0.45\textwidth}
%		\centering
%		\includegraphics[scale=.4]{fig/prime_unsolved_pie.pdf}
%	\end{subfigure}
%	\caption{Pie charts for the neighborhood size of solved and unsolved presentations}
%	\label{fig:prime_pies}
%\end{figure}

As we saw, having maximal neighborhood size provides significant information about the solved/unsolved label of a presentation.
Additionally, the minimum neighborhood size among unsolved presentations --89,573-- is also quite informative since 54\% of solved presentations have neighborhood sizes less than that.
We can improve this percentage using that neighborhood sizes of unsolved presentations concentrate in three bands.
Please consult \cref{fig:prime_histogram}.
We have that 64.3\% of the solved presentations lie outside of the three bands $[89575, 89575]\ \& \ [89715, 89831]\ \& \ [89844, 89872]$ which contain over 99\% of the unsolved presentations.

By replacing the last band with $[89859,89872]$, the union of these three bands now encompasses the neighborhood sizes of over 90\% of unsolved presentations, while its complement includes 77.2\% of those associated with solved presentations.

\begin{figure}
	\centering
    \includegraphics[scale=.34]{fig/prime_histogram_rl.pdf}
	\includegraphics[scale=.34]{fig/prime_histogram_rl2.pdf}
	\caption{Histograms with 6 and 26 bins respectively of the neighborhood sizes of the 417 PPO-solved and 773 PPO-unsolved presentations.}
	\label{fig:prime_histogram}
\end{figure}

\medskip

One might anticipate that enhancing the discriminatory power of $n$-neighborhoods could be achieved by incorporating features beyond their size. We explored two additional types of features, but surprisingly, they only marginally improved the accuracy of solved/unsolved predictions. The first type was based on node centrality, while the second focused on spectral features of the neighborhood graphs. The latter was particularly intriguing, considering the emphasis on Markov processes and the well-known relationship between random walks on graphs and the graph Laplacian.