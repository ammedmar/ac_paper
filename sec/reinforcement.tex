% !TEX root = ../ac_paper.tex

\section{Reinforcement Learning}
Why RL and not BFS? 1. memory efficient, 2. parallelizable/distributed.
\subsection{Markov Decision Process}

We will model the problem underlying the Andrews-Curtis conjecture as a Markov Decision Process (MDP).
An MDP is a tuple of data $(S, A, P, R, \rho)$ where
\begin{itemize}
	\item $S$ is a set of states called the state space.
	\item $A$ is a set of actions that acts on $S$, called the action space.
	That is, each element $a \in A$ is a map $a \colon S \to S$.
	\item $P \colon S \times A \times S \to \mathbb{R}$ is the transition
	probability function.
	It is the probability $p (s' | a, s)$ of reaching
	state $s'$ as we take action $a$ in state $s$.
	\item $R \colon S \times A \times S \to \mathbb{R}$ is the 'reward' function.
	\item $\rho$ is the initial probability distribution of states.
\end{itemize}

In the setting of the Andrews-Curtis conjecture, $S$ is the set of all presentations of the trivial group, and $A$ is the set of AC moves.
The transition probability function $P$ is the probability of applying a particular AC move to a given presentation.
The choice of reward function $R$ is up to us.
One suitable option (the only one we have worked with so far) is to take $R(s)$ for a presentation $s$ to be the negative of the total word length ---i.e., the sum of word lengths of each relator in the presentation.
(AK(3) has a total word length of 13.)
This choice depends only on the final state $s'$ and is independent of the initial state $s$
or the action $a$.
\footnote{Another good choice could be the difference in total word lengths of the final and the initial states.} It is suitable for the goal of trivializing a presentation as the trivial presentation has the maximum possible reward value: negative of the number of generators.

Differences in our AC moves and the usual notion of AC moves: 1.
since we restrict lengths, our AC moves are not invertible; 2.
we use full-simplify which cyclically reduces words before returning the answer.
For example, if we conjugate a word and it takes the form $y \cdots y^{-1}$, our conjugation operation returns $\cdots$.


\subsection{Proximal Policy Optimization}


\subsection{Results}

\begin{enumerate}
\item Shorter sequences of AC moves. 
\end{enumerate}
