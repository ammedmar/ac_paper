% !TEX root = ../ac_paper.tex

\section{Stable Andrews-Curtis conjecture\label{sec:SAC}}

There also exists a weaker version of the Andrews-Curtis conjecture, also known as the "stable Andrews-Curtis conjecture".
It allows, in addition to the AC moves presented above, a "stability move" --- i.e., extend a presentation $\angles{x_1, \dots, x_n \mid r_1, \dots, r_n}$ to include more generators:
\[
\angles{x_1, \dots, x_n, x_{n+1}, \dots x_m \mid r_1, \dots, r_n, x_{n+1}, \dots, x_m}.
\]

It is not known if stable AC-equivalence between two presentations implies AC-equivalence between the two presentations.
A potential counterexample to this statement is given in \cite{MMS}: the presentation $\langle x, y \mid x^4 = y x^2 y^{-1} x^{-1} y x^2 y^{-1}, y = [x^2, y]^3 $ of total word length 32 is stably AC-equivalent to $\AK(3)$ but it is not known if the two presentations are AC-equivalent.
\footnote{Actually, now that we know that AK(3) is stably AC-trivial, maybe we should check if this length 32 presentation is AC-trivial or not.}
\footnote{Exploring this presentation with our RL agent could be useful in showing that the two are indeed equivalent.}
\newline