% !TEX root = ../ac_paper.tex

\subsection{Length reduction for $\AK(n)$}

Using the greedy search algorithm, we found a reduction of $\AK(5)$ to length 16, using [resources, a sequence total relator length maxxes out at 25].
It was previously unknown whether any length reduction was possible for any of the presentations in the $\AK(n)$ sequence for $n \geq 3$.

By examining this reduction, we show that it can be generalized to show that $\AK(n)$, which has total relator length $2n+7$, is AC-equivalent to a presentation of length $n+11$, which is a reduction in length for every $n \geq 5$.
Explicitly, we will prove \autoref{t:n+11} stating that $\AK(n)$ is AC-equivalent to the presentation
\[
\langle x,y \mid xyx=yx^{n-1}y, y^2x=xyx^{-1}y \rangle.
\]

To do so we will prove the following two lemmas from which the theorem follows easily.
In both lemmas we will use the following definition
\[
P(k) = \langle x,y \mid xyx=yx^k y, y^{-1}x^n y = x^{n+1}\rangle.
\]

\begin{lemma}
    For any $k<n$, $\AK(n)$ is AC-equivalent to $P(k)$.
\end{lemma}

\begin{proof}
    In \cite{MMS}, it is shown that $AK(n)$ is AC-equivalent to $P(1)$. Thus, we show that for any $k<n-1$, $P(k)$ is AC-equivalent to $P(k+1)$, and the result follows by induction.

    First, note that second relation in $P(k)$ can be rearranged in the following three ways:
    \begin{enumerate}[label=(\roman*)]
        \item $x^kyx^{-1}=x^{k-n}yx^n$
        \item $xyx^{-n}=x^{-(n-1)}yx$
        \item $x^{n-k}y^{-1}x^{-(n-1)}=x^{-(k+1)}y^{-1}x$
    \end{enumerate}

    We can rearrange the first relation of $P(k)$ to get $x^k y x^{-1}=y^{-1}xy$ and substitute from (i) to get $x^{k-n} y x^n = y^{-1}xy$. Then we rewrite this as $xyx^{-n}=yx^{k-n}y$ and substitute from (ii) to get $ x^{-(n-1)}yx= yx^{k-n}y$. Finally, we rewrite this as $x^{n-k}y^{-1}x^{-(n-1)}=yx^{-1}y^{-1}$ and substitute from (iii), which gives $x^{-(k+1)}y^{-1}x=yx^{-1}y^{-1}$ or equivalently $xyx=yx^{k+1}y$.
\end{proof}

\begin{lemma}
    $P(n-1)$ is AC-equivalent to $\langle xyx=yx^{n-1}y, y^2x=xyx^{-1}y \rangle$
\end{lemma}

\begin{proof}
    The first relation of $P(n-1)=\langle x,y \mid xyx=yx^{n-1} y, y^{-1}x^n y = x^{n+1}\rangle$ can be equivalently expressed as
    \begin{enumerate}[label=(\roman*)]
        \item $x^{n-1}yx^{-1}=y^{-1}xy$
        \item $x^{-1}yx^{n-1}=yxy^{-1}$
    \end{enumerate}
    Writing the second relation of $P(n-1)$ as  $(x^{n-1}yx^{-1})x^{-1}=x^{-1}yx^{n-1}$, we substitute from (i) on the LHS and (ii) on the RHS. This gives $y^{-1}xyx^{-1}=yxy^{-1}$, which we can rearrange to get the result.
\end{proof}

