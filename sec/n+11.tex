% !TEX root = ../ac_paper.tex

\subsection{Length reduction for $\AK(n)$}

Using the greedy search algorithm, we found a reduction of $\AK(5)$ to length $16$, using a little less than $20$ million nodes.
It was previously unknown whether any length reduction was possible for any of the presentations in the $\AK(n)$ sequence for $n \geq 3$.

By examining this reduction, we find that it can be generalized to show that $\AK(n)$, which has total relator length $2n+7$, is AC-equivalent to a presentation of length $n+11$. This is a reduction in length for every $n \geq 5$.
Explicitly, we will prove \autoref{t:n+11} stating that $\AK(n)$ is AC-equivalent to the presentation
\[
\langle x,y \mid y^{-1} x^2 y = x y^{-1} x \ , \ y x y = x y^{n-1} x \rangle.
\]

To do so we will prove the following two lemmas from which the theorem follows easily.

\begin{lemma}
    For any $k<n$, $\AK(n)$ is AC-equivalent to $MS(n, w_k)$ where $w_k = y^{-k} x^{-1} y x y$. 
    \todo{Is it clear that k must be less than n? Also can k be zero or negative? If k can be zero or negative, then does the induction argument below need a modification? Also, if k does not need to be bound by n, then we have obtained two infinite families of AC-trivial presentations, i.e. $MS(1, w_k)$ and $MS(2, w_k)$. I think that we don't need k < n bound. If we do, then we should remove the remark below.}
    \[
\MS(n, w_k) = \angles{x, y \mid x^{-1} y^n x = y^{n+1}, x = y^{-k} x^{-1} y x y}.
\]
\end{lemma}


\begin{proof}
    In \cite{MMS}, it is shown that $AK(n)$ is AC-equivalent to $MS(n, w_1)$. Thus, we show that for any $k<n-1$, $MS(n, w_k)$ is AC-equivalent to $MS(n, w_{k+1})$, and the result follows by induction.

    First, note that the first relation in $MS(n, w_k)$ can be rearranged in the following three ways:
    \begin{enumerate}[label=(\roman*)]
        \item $y^kxy^{-1}=y^{k-n}xy^n$
        \item $yxy^{-n}=y^{-(n-1)}xy$
        \item $y^{n-k}x^{-1}y^{-(n-1)}=y^{-(k+1)}x^{-1}y$
    \end{enumerate}

    Now, we can rearrange the relation $x=w_k$ to get $y^k x y^{-1}=x^{-1}yx$ and substitute from (i) to get $y^{k-n} x y^n = x^{-1}yx$. Then we rewrite this as $yxy^{-n}=xy^{k-n}x$ and substitute from (ii) to get $ y^{-(n-1)}xy= xy^{k-n}x$. Finally, we rewrite this as $y^{n-k}x^{-1}y^{-(n-1)}=xy^{-1}x^{-1}$ and substitute from (iii), which gives $y^{-(k+1)}x^{-1}y=xy^{-1}x^{-1}$ or equivalently $x=y^{-(k+1)}x^{-1}yxy=w_{k+1}$. 
\end{proof}

\begin{lemma}
    $MS(n, w_{n-1})$ is AC-equivalent to $\langle x,y \mid y^{-1} x^2 y = x y^{-1} x \ , \ y x y = x y^{n-1} x \rangle$.
\end{lemma}

\begin{proof}
    The relation $x=w_{n-1}$ in $MS(n, w_{n-1})$ can be equivalently expressed as
    \begin{enumerate}[label=(\roman*)]
        \item $y^{n-1}xy^{-1}=x^{-1}yx$
        \item $y^{-1}xy^{n-1}=xyx^{-1}$
    \end{enumerate}
    Writing the first relation of $MS(n, w_{n-1})$ as  $(y^{n-1}xy^{-1})y^{-1}=y^{-1}xy^{n-1}$, we substitute from (i) on the LHS and (ii) on the RHS. This gives $x^{-1}yxy^{-1}=xyx^{-1}$, which we can rearrange to get the result.
\end{proof}

\begin{remark}
As a corollary of Lemma 4, $MS(n, w_k)$ for each fixed $n$ is an infinite family of AC-equivalent presentations. As $\AK(1)$ and $\AK(2)$ are known to be AC-trivial, $MS(1, w_k)$ and $MS(2, w_k)$ are infinite families of AC-trivial presentations.
\end{remark}

\begin{remark}
In Lemma 5, we obtained a presentation that is AC equivalent to $MS(n, w_{n-1})$. It is possible to find various other length reductions of $AK(n)$ by applying the procedure in the proof of Lemma 5 to $MS(n, w_k)$ for $k < n-1$. 
\end{remark}

