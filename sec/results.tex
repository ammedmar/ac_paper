% !TEX root = ../ac_paper.tex

\section{Results and current challenges}

Why RL and not BFS? 1. memory efficient, 2. parallelizable/distributed.
As a consequence,  we can achieve higher bounds on the length of sequence of AC moves that we can apply on a presentation.
Figure 1 in Bowman-McCaul seems to show maximum depth to be around 40.
That is, at best, we can search through sequences of 40 AC moves with BFS.


https://arxiv.org/pdf/2110.00641.pdf

What is the limit on maximum path length? What's the limit achieved by previous BFS papers?
\fixme{Things to try: change the initial state distribution in RL to include canonic presentations of certain lengths.}

\fixme{If we have a "good" policy for a small max length $k$, a new "good" policy for a larger max length $k'$ should do one of two things things: i).
for state $s$ with length $< k$ and with $a(s) $ also of length $< k$ for all $a$, $\pi(s)$ should be the same; ii) learn new optimal policy i.e. new $\pi(s)$ for all other $s$ starting with random policy.
This kind of states include states of length $> k$ as well as states of length $< k$ for which $a(s) $ has length $> k$ for some $a$.}

\fixme{When we increase max length, value of states of type i) could still change.
We could replace the condition in i) by the condition that states $s$ are trivializable.
but even then, their values could change.
In particular, an optimal policy function might become sub-optimal (restricted to states of smaller length) when max length is increased.}

\fixme{Compare the performance with an ordinary BFS algorithm. I set it up to run on Feb 23 with SLURM ID: 34869100. Compare it also with a GS/BFS algorithm that applies the usual AC moves. 34869254 has GS with original AC moves and 34869256 has BFS with original AC moves. 
Update: there was an issue with 34869254. I submitted a new run 34943081 after fixing the issue. }


\subsection{Future directions of RL}
Scaling?

\subsection{Potential counterexamples of AC}

\subsubsection{Miller-Schupp series}

With the help of Breadth-first search and PPO, we have been able to show that all examples of Miller-Schupp series of total word length 14 are equivalent to either trivial presentation or $\AK(3)$, with the following two exceptions:
\[
\angles{x, y \mid x^{-1} y^2 x = y^{3} , x = y x^2 y^{\pm 1} x^{-2}}
\]
It is listed as example 1 on page 3 of \cite{MMS}.
I have checked whether we can reduce the length of these examples with max length = 20.

\fixme{Compare with Table 2 of Panteleev-Ushakov.
They seem to have a lot more examples with $n=2$ that they could not resolve.}

\subsubsection{Gordon's series}

\subsection{Does stable AC equivalence imply AC equivalence?}

In \cite{MMS}, the authors give examples of stably AC-equivalent pairs of presentations for which it is not known whether they are also AC-equivalent.
We consider one of these examples, which is stably AC-equivalent to the trivial presentation:
\[
\angles{x, y \mid xyx^{-2}y^{-1} xy^{-1}, x^{-1} y^{-1} x y^2 x y^{-1}}.
\]

We have shown that it is AC-equivalent to $\AK(3)$.
This implies that $\AK(3)$ is stably AC-equivalent to the trivial presentation.

\fixme{This is AC-equivalent to $[ 2  2 -1 -1 -1  2  2  0  0  0  0  0  0  0  0  0  0  0  0  0 -1 -2 -1  2
	1  2  0  0  0  0  0  0  0  0  0  0  0  0  0  0]$ as shown by $mms_to_ak3.py$ around update 99.

	$x^2 y^{-3} x^2 = 1 \to x^2 y^3 = x^{-2} \to x^{4} y^3 =1 \to y^3 x^{-4} =1 \to y^3 = x^4 $ and $x^{-1} y^{-1} x^{-1} y x y = 1 \to x y x = yxy$.}

We can relate the length-25 example of MMS to AK(3) in 53 moves: $[(9, 25), (7, 25), (4, 17), (8, 17), (11, 17), (5, 17), (11, 17), (9, 17), (3, 16), (10, 16), (12, 16), (7, 16), (7, 16), (9, 16), (11, 16), (5, 16), (3, 15), (5, 15), (4, 19), (3, 14), (12, 14), (5, 14), (7, 16), (7, 18), (1, 19), (9, 19), (11, 19), (8, 19), (3, 18), (5, 18), (10, 18), (2, 15), (6, 15), (12, 15), (9, 15), (7, 15), (5, 15), (11, 15), (10, 15), (3, 15), (8, 15), (11, 15), (9, 15), (2, 16), (10, 16), (12, 16), (5, 16), (7, 16), (9, 16), (11, 16), (1, 13), (9, 13), (8, 13)]$.

The best BFS implementation that works is in src/breadth-first-search/bfs.py.

I need to change -- to lengths in ACMove in my implementations of ACMove.

This also implies that the simplest example of a presentation that satisfies the weak AC conjecture but not the AC conjecture is AK(3).

It would be nice if we can also solve AK(4).
Even if it happens through a computer program involving BFS.

I should look at the results of my attempt to trivialize MMS-proposition-1.2 example on W\&B.

This is one of those examples that satisfies the weak AC conjecture but was not known to satisfy the AC conjecture.
If it is equivalent to trivial instead of AK(3), we have obtained a new result.
One question though: in the paper, they obtain this presentation from a rank-3 presentation after substitution.
But is substitution a valid AC move (or a sequence of it)?

Note that Theorem 1.6 of MMS says that there exist stably AC-inequivalent but AC-equivalent 2-element sets.

" It seems to
be another very difficult question as to whether or not there are stably ACequivalent, but AC-inequivalent sets of the same cardinality in a free group." [MMS]
