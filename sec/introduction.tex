% !TEX root = ../ac_paper.tex

\section{Introduction\label{sec:intro}}

We live in an extraordinary era where artificial intelligence (AI) is transforming numerous sectors and professions. Recent advancements in Large Language Models (LLMs) have empowered AI to read, write, and converse with a proficiency comparable to that of human experts. In the realm of board games, AI has outperformed even the most skilled human players, and it has tackled complex scientific challenges like protein folding, where steady progress was suddenly overtaken by a near-complete solution. As AI continues to evolve, one critical question remains: How wide is the range of domains in which AI systems can reason as effectively as humans?

Mathematics appears to be a natural progression on the path toward Artificial General Intelligence (AGI) due to its universal syntactic and logical structure, similar to that of natural language. Additionally, mathematics provides a framework for the quantitative evaluation of logical and analytical reasoning, making it an ideal domain for self-improving AI systems on the path to AGI. In a moment, we will explain another reason why mathematics could play a crucial role in AGI development, but first, we need to introduce one more key element: reinforcement learning (RL).

Machine learning, a subfield of AI, involves developing algorithms and statistical models that enable computers to learn from data and make predictions. Among the three primary areas of machine learning --- supervised learning, unsupervised learning, and reinforcement learning --- RL emphasizes learning through interaction with an environment and receiving feedback in the form of rewards or penalties. This aspect of machine learning, often characterized by its focus on AI models `playing games,’ will be central to our discussion.

A typical chess game lasts about 30 to 40 moves, with the longest recorded professional game reaching 269 moves, ending in a draw between Ivan Nikolic and Goran Arsovic in 1989. Notably, the number of moves in a typical chess game is relatively consistent, with the longest professional game having only about an order of magnitude more moves than the average. Similarly, a typical game of Go involves a few hundred moves, with the longest recorded professional game, played by Go Seigen and Kitani Minoru in 1933, lasting 411 moves.

At first glance, proving or disproving mathematical conjectures can be formulated as games. For example, proving a theorem involves finding a path from the hypothesis to the conclusion, consisting of basic logical steps, such as Lean steps. From the RL perspective, this type of game can be quite complex due to its large action space. Conversely, finding examples or counterexamples to settle important conjectures may require only a few basic moves (actions); the case study in this paper serves as a good illustration of such a problem. In all cases, the problem is fundamentally a search process, with complexity largely determined by the size of the action space and the search space.

In addition, with hard research-level mathematics problems, one faces yet another challenge: the sought-after instance can be so rare and difficult to find that the problem effectively becomes a search for a needle in a haystack, i.e., a problem with ultra-sparse rewards. For example, in the context of theorem proving, one might consider an extremely hard theorem\footnote{A proxy for such a problem could be the Riemann Hypothesis or any other unsolved Millennium Prize Problem.} that may require a very large number of steps. If there aren't many alternative proofs, finding even a small number of very long ones then becomes akin to searching for a needle in a haystack or, depending on one's preference, a search for a unicorn.\footnote{Similar challenges, though not as critical, also arise in non-research-level math problems; see, e.g., \cite{peano,dabelow2024symbolicequationsolvingreinforcement,trinh2024solving} for recent discussion. Meanwhile, in a parallel line of development, new benchmarks have been proposed in the past couple of years \cite{procgen,NeedleInAHaystack}, which can be useful in such contexts.}

Fortunately, mathematics offers a robust framework for developing and testing new algorithms with adaptive capabilities that dynamically `learn how to learn.' Testing these algorithms on mathematical problems, rather than directly on societal issues like market crash predictions or extreme weather events, provides a risk-free and cost-effective approach. Additionally, this method offers the dual benefit of potentially solving hard mathematical problems and resolving long-standing conjectures in the process.

Our approach to problems of this type involves equipping the RL model with the ability to assess the hardness of problems during the training process. First and foremost, this requires a practically useful notion of hardness, a concept we thoroughly explore. By learning the distribution of problems based on their difficulty, one can enhance existing off-the-shelf algorithms with new self-improvement strategies, identifying key features that facilitate solving the most challenging problems.

In this paper, we begin a systematic implementation of this approach by carefully analyzing the distribution of hardness in potential counterexamples to a long-standing conjecture, the Andrews--Curtis conjecture. As with many other challenging problems, a natural measure of hardness is the number of steps an RL agent needs to take. What makes the Andrews--Curtis conjecture particularly suitable for our study is that it includes numerous examples requiring a hyper-exponential number of steps, providing an effective testbed for exploring the aforementioned questions through the lens of RL, search algorithms, and topological data analysis.


While our primary focus is on exploring the distribution of hardness to inform algorithm development, we also discover new results about the long-standing potential counterexamples of this conjecture. A notable family of potential counterexamples, denoted $\AK(n)$, due to Akbulut and Kirby, 
\[
\AK(n) = \angles{x, y \mid x^n = y^{n+1}, xyx = yxy}, \quad n \geq 3
\]
has been open in mathematics for more than four decades \cite{Akbulut--Kirby}.
\footnote{The original work included the case $n=2$, but this case was solved approximately two decades later \cite{genetic}. The case $n=1$ also gives a balanced presentation of the trivial group, which is easy to solve in a few steps.}
The length of the presentation $\AK(n)$ is $2n + 7$, and 
until now, it was unknown whether this length can even be reduced using AC moves. We find that for every fixed $n$, $\AK(n)$ is AC-equivalent to each element of the following 1-parameter family of presentations parameterized by $k \in \mathbb{Z}$,
\[P(n, k) = \angles{x, y \mid  y^{n-k-1}  x^{-1} y x = x y x^{-1} y^{n-k} \ , \ x = y^{-k} x^{-1} y x y }.
\]
When $n \geq 5$, this family contains presentations of length less than $2n + 7$. The shortest presentation occurs in the case $k=n-1$, which gives us the following theorem.

\begin{theorem}\label{t:n+11}
	For every $n\geq 2$, $\AK(n)$ is AC-equivalent to the presentation
	\[
	\angles{ x,y \mid x^{-1} y x = x y x^{-1} y \ ,\  xyx=yx^{n-1}y },
	\]
	of length $n+11$. This gives a reduction in length of $AK(n)$ for all $n \geq 5$.
\end{theorem}

Another important family of potential counterexamples due to Miller and Schupp,
\[
\MS(n, w) = \angles{x, y \mid x^{-1} y^n x = y^{n+1}, x = w},
\]
where $n > 0$ and $w$ is a word with exponent sum zero on $x$, has been open for more than 25 years \cite{Miller--Schupp}. We obtain the following result.

\begin{theorem}\label{t:msfamilies}
        The following subfamilies of Miller-Schupp presentations are AC-trivial:
        \begin{enumerate}[label=(\roman*)]
            \item $\MS(1, w)$ for all $w$,
            \item $\MS(n, w_\star)$ for $w_\star = y^{-1} x y x^{-1}$,
            \item $\MS(2, w_k)$ for $w_k = y^{-k} x^{-1} y x y$ where $k \in \mathbb{Z}$.
        \end{enumerate}
        Furthermore, for each fixed $n > 0$, all members of the 1-parameter family, $\MS(n, w_k)$, parameterized by $k \in \mathbb{Z}$, are AC-equivalent to each other and to $\AK(n)$.
\end{theorem}

In addition to these results involving infinite families, we also found AC-trivializations of hundreds of Miller-Schupp presentations. Of special note are two hard presentations,
which our classical search algorithms could not solve at first. We trained reinforcement learning agents to find,

\begin{theorem}\label{thm:MS}
    The following potential counterexamples belonging to the Miller--Schupp series are AC-trivial:
	\begin{gather*}
		\angles{x, y \mid x^{-1} y^3 x y^{-4} , \ x^{-1} y^{-1} x y^{-3} x^{-1} y^{-1}}, \\
		\angles{x, y \mid x^{-1} y^4 x y^{-5} , \ x^{-1} y x y^{-1} x^{-1} y^{-3}}.
	\end{gather*}
\end{theorem}

Note that the first of these presentations is extremely similar to the following rewriting of $\MS(3, w_{-3})$,
\[
\MS(3, w_{-3}) = \angles{x, y \mid x^{-1} y^3 x y^{-4}, x^{-1} y^{-1} x y^{-3} x y^{-1}}.
\]
with all but one letter being different. The latter presentations is AC-equivalent to $\AK(3)$ (from \cref{t:msfamilies}) --- perhaps the most important potential counterexample to the Andrews-Curtis conjecture.
This shows how very similar-looking presentations can have vastly different hardness properties, emphasizing the need for a systematic understanding of hardness in math problems.

\medskip
A weaker version of the AC conjecture can be obtained by incorporating additional \textit{stable moves}. This establishes a direct connection between the conjecture and knot theory, a relationship we explore in \cref{sec:stable}. Motivated by this connection and by the claim in \cite{MMS} that the presentation
\begin{equation}\label{eq:MMS3}
	\langle x, y \mid
	x^{-1}y^{-1}xy^{-1}x^{-1}yxy^{-2}xyx^{-1}y, \
	y^{-1}x^{-1}y^2x^{-1}y^{-1}xyxy^{-2}x \rangle
\end{equation}
is stably AC-trivial, we apply reinforcement learning to demonstrate its AC equivalence to $\AK(3)$.
However, after closer inspection, we discovered a subtle misprint in \cite[p.10]{MMS} that had gone unnoticed for more than 20 years. This misprint undermines the claim that \eqref{eq:MMS3} is stably AC-trivial.\anibal{@SB: Maybe worth mentioning the heuristics that lead the SB side of the team to wonder about the result in the first place. I seem to recall that certain complexity considerations motivated the closer look.}

\medskip
We should emphasize that this entire project was carried out using relatively modest computational resources that a small academic research group can easily afford. Consequently, we placed greater emphasis on theoretical tools and techniques that, when combined with scaled-up computational resources, can lead to further advancements.




\medskip The rest of the paper is organized as follows. In \cref{sec:AC}, we provide an overview of the Andrews--Curtis conjecture. We then apply classical search algorithms to examples in the Miller--Schupp series in \cref{sec:search}, where we devise a greedy search algorithm that significantly outperforms the widely used breadth-first search algorithm. In \cref{sec:rl}, we employ reinforcement learning, specifically the Proximal Policy Optimization (PPO) algorithm \cite{schulman2017proximal}, and find that while it performs better than breadth-first search and solves some presentations that hard for greedy search, it does not strictly surpass the greedy search algorithm.
%(see \cref{fig:performance})
Building on these insights, \cref{sec:algo} presents several ideas for algorithm development, proposing strategies to mitigate the overwhelming complexity faced by an RL agent in challenging problems.
In \cref{sec:lm}, we examine the linguistic structure of balanced presentations using a decoder-only Transformer model, observing distinct clusters within the embedding space corresponding to presentations solved and unsolved by the greedy search algorithm.
In \cref{sec:isolated}, we employ topological methods to assess the complexity of presentations. Specifically, we utilize persistent homology to define the \textit{isolation value} of a presentation. This value is infinite for counterexamples to the AC conjecture and serves as a measure of a presentation's resistance to being trivialized. Finally, in \cref{sec:stable}, we study the stable version of the Andrews-Curtis conjecture, and its relationship with knot theory.
This relationship has previously been used to give potential counterexamples of the standard Andrews--Curtis conjecture. We prove AC-triviality of a large class of these presentations. 
% and propose that the rest are also all AC-trivializable. 

\medskip\noindent
We encourage the reader to explore the accompanying GitHub repository:
\begin{center}
	\href{https://github.com/shehper/AC-Solver}{https://github.com/shehper/AC-Solver}\todo{Add URL to the comments section of the arXiv's metadata.}
\end{center}