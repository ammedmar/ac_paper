% !TEX root = ../ac_paper.tex

\section{Future Directions}

Why RL and not BFS? 1. memory efficient, 2. parallelizable/distributed.
As a consequence,  we can achieve higher bounds on the length of sequence of AC moves that we can apply on a presentation.
Figure 1 in Bowman-McCaul seems to show maximum depth to be around 40.
That is, at best, we can search through sequences of 40 AC moves with BFS.

What is the limit on maximum path length? What's the limit achieved by 
previous BFS papers?

Change the initial state distribution in RL to include canonic presentations of certain lengths.

Scale RL?

Apply to other types of series such as Gordon's series?

Does stable AC equivalence imply AC equivalence? Explore this more.

It would be nice if we can also solve AK(4) for stable AC-equivalence.


I should look at the results of my attempt to trivialize MMS-proposition-1.2 example on W\&B.

Note that Theorem 1.6 of MMS says that there exist stably AC-inequivalent but AC-equivalent 2-element sets. Understand that better.

There is also a length-14 presentation in \cite{MMS} for which the authors could not reduce the length of the relators using AC moves.
\[
\angles{x, y \mid xyx^{-2}y^{-1} xy^{-1}, x^{-1} y^{-1} x y^2 x y^{-1}}.
\]

People have noted before that the following presentation is AC-trivial. One such element is the case $n=3$, $w = y^{-1} x y x^{-1}$; this example is AC-trivializable \cite{morse}. Did we find it to be AC-trivial through GS?

There also exist other generalizations of the conjecture for other kinds of groups; some of which have been proved.
In particular, the conjecture holds true for finite groups and soluble groups \cite{Borovik, Guyot}.
I think these versions will not be important for us but can one learn algorithmic lessons from applying RL / GS to those cases?

 
 This paper says that either the Andrews Curtis conjecture is false or there is an algorithm to recognise balanced presentations of the trivial group: https://arxiv.org/pdf/math/0108053.pdf

