\subsection{New trivializations in the MS series}\label{sec:RLsolutions}

The goal of this section is to provide explicit descriptions of the sequences of AC moves identified by our RL agent, which establish the AC-triviality of the following MS presentations (\autoref{thm:MS}):
\begin{gather*}
    \angles{x, y \mid x^{-1} y^3 x y^{-4} , \ x^{-1} y^{-1} x^{-1} y^{-1} x y^{-3}}, \\
    \angles{x, y \mid x^{-1} y^4 x y^{-5} , \ x^{-1} y x y^{-1} x^{-1} y^{-3}}.
\end{gather*}
First, it is important to highlight a general aspect relevant to any RL-assisted research in mathematics: While RL models are inherently stochastic, with their performance influenced by several factors discussed here and in \autoref{sec:algo}---including algorithm choice, data representation, and architecture---the solution, once identified, is fully deterministic and rigorous. In other words, using RL to solve complex mathematical problems is similar to solving Sudoku puzzles: finding a solution can be highly challenging, but verifying its correctness is straightforward and fast (verifiable in polynomial time).

Thus, while the earlier part of this section focused on the process of finding solutions, our goal here is to present rigorous solutions (sequences of AC moves) as discovered by the RL agent.

For the two presentations above, the AC paths identified by the agent had lengths of $205$ and $1085$, respectively.
In the first presentation, the initial sequence length was $17$, which increased to $29$ along the AC path before the agent reached the trivial final state. Similarly, in the second presentation, the initial length was $19$, which increased to $30$ before the agent completed the trivialization.

Starting with these AC paths, we further simplified them, ultimately reducing their lengths to $195$ and $381$ moves, respectively.\anibal{@Shehper: Did this keep the maximum length in the path?}

The length 195 path:\anibal{@anyone: Please explain what the $h_i$ are}
\[
\begin{aligned}
	& h_{2} \cdot h_{8} \cdot h_{10} \cdot h_{4} \cdot h_{10} \cdot h_{10} \cdot h_{10} \cdot h_{2} \cdot h_{8} \cdot h_{6} \cdot h_{6} \cdot h_{6} \cdot h_{8} \cdot h_{10} \cdot h_{10} \cdot h_{10} \cdot h_{10} \cdot h_{4} \\ &
	h_{0} \cdot h_{6} \cdot h_{2} \cdot h_{8} \cdot h_{8} \cdot h_{10} \cdot h_{4} \cdot h_{0} \cdot h_{6} \cdot h_{8} \cdot h_{10} \cdot h_{10} \cdot h_{10} \cdot h_{2} \cdot h_{8} \cdot h_{6} \cdot h_{6} \cdot h_{4} \\ &
	h_{10} \cdot h_{10} \cdot h_{10} \cdot h_{2} \cdot h_{8} \cdot h_{6} \cdot h_{8} \cdot h_{10} \cdot h_{4} \cdot h_{0} \cdot h_{6} \cdot h_{8} \cdot h_{10} \cdot h_{10} \cdot h_{2} \cdot h_{8} \cdot h_{6} \cdot h_{6} \\ &
	h_{4} \cdot h_{10} \cdot h_{10} \cdot h_{10} \cdot h_{2} \cdot h_{8} \cdot h_{6} \cdot h_{6} \cdot h_{8} \cdot h_{10} \cdot h_{4} \cdot h_{0} \cdot h_{6} \cdot h_{8} \cdot h_{10} \cdot h_{2} \cdot h_{8} \cdot h_{6} \\ &
	h_{6} \cdot h_{4} \cdot h_{10} \cdot h_{10} \cdot h_{10} \cdot h_{2} \cdot h_{8} \cdot h_{6} \cdot h_{6} \cdot h_{6} \cdot h_{8} \cdot h_{10} \cdot h_{4} \cdot h_{0} \cdot h_{6} \cdot h_{8} \cdot h_{2} \cdot h_{8} \\ &
	h_{6} \cdot h_{6} \cdot h_{4} \cdot h_{10} \cdot h_{10} \cdot h_{10} \cdot h_{1} \cdot h_{7} \cdot h_{5} \cdot h_{5} \cdot h_{5} \cdot h_{11} \cdot h_{9} \cdot h_{2} \cdot h_{8} \cdot h_{6} \cdot h_{4} \cdot h_{2} \\ &
	h_{8} \cdot h_{6} \cdot h_{4} \cdot h_{2} \cdot h_{8} \cdot h_{6} \cdot h_{6} \cdot h_{8} \cdot h_{10} \cdot h_{4} \cdot h_{10} \cdot h_{10} \cdot h_{2} \cdot h_{8} \cdot h_{6} \cdot h_{6} \cdot h_{2} \\ &
	h_{8} \cdot h_{6} \cdot h_{6} \cdot h_{8} \cdot h_{0} \cdot h_{6} \cdot h_{8} \cdot h_{6} \cdot h_{8} \cdot h_{10} \cdot h_{4} \cdot h_{10} \cdot h_{8} \cdot h_{6} \cdot h_{2} \cdot h_{8} \cdot h_{6} \\ &
	h_{6} \cdot h_{8} \cdot h_{0} \cdot h_{6} \cdot h_{8} \cdot h_{6} \cdot h_{8} \cdot h_{10} \cdot h_{2} \cdot h_{8} \cdot h_{6} \cdot h_{6} \cdot h_{8} \cdot h_{0} \cdot h_{6} \cdot h_{8} \cdot h_{6} \\ &
	h_{8} \cdot h_{8} \cdot h_{6} \cdot h_{8} \cdot h_{0} \cdot h_{6} \cdot h_{8} \cdot h_{6} \cdot h_{8} \cdot h_{8} \cdot h_{0} \cdot h_{6} \cdot h_{8} \cdot h_{6} \cdot h_{8} \cdot h_{10} \cdot h_{0} \\ &
	h_{6} \cdot h_{8} \cdot h_{6} \cdot h_{8} \cdot h_{8} \cdot h_{1} \cdot h_{7} \cdot h_{5} \cdot h_{11} \cdot h_{9} \cdot h_{3} \cdot h_{11} \cdot h_{9} \cdot h_{11} \cdot h_{8} \cdot h_{0} \cdot h_{6} \cdot h_{3} \cdot h_{11}.
\end{aligned}
\]

The length 381 path is presented in \autoref{app:RLpath}.

Note that, with $12$ basic moves, the search space of such magnitude has $12^{1546} \sim 10^{1668}$ states, which is much larger than googol ($10^{100}$).