% !TEX root = ../ac_paper.tex

\begin{abstract}
    Using a long-standing conjecture from combinatorial group theory, we explore, from multiple angles, the challenges of finding rare instances carrying disproportionately high rewards. Based on lessons learned in the mathematical context defined by the Andrews--Curtis conjecture, we propose algorithmic improvements that can be relevant in other domains with ultra-sparse reward problems.
    In the course of this study, we also address several open mathematical questions. Specifically, we establish the AC-triviality of two potential counterexamples from the Miller--Schupp family (1991) and demonstrate the AC length reducibility of all but two presentations in the Akbulut--Kirby (infinite) family (1981).
\end{abstract}

% \begin{abstract}
% 	Using a long-standing conjecture from combinatorial group theory, we explore, from multiple angles, the challenges of finding rare instances carrying disproportionately high rewards. Based on lessons learned in the mathematical context defined by the Andrews--Curtis conjecture, we propose algorithmic improvements that can be relevant in other domains with ultra-sparse reward problems.
%     Although our case study can be formulated as a game, its shortest winning sequences are potentially $10^6$ or $10^9$ times longer than those encountered in chess.
%     In the course of our study, we establish the AC-equivalence or AC-triviality of several significant potential counterexamples, including those from the Miller--Schupp family (1991) and the examples proposed by Akbulut and Kirby (1981).
% 	%whose status escaped direct mathematical methods for decades.
% 	%whose status had resisted direct mathematical methods for decades.
% 	%whose status had resisted resolution through direct mathematical methods for decades.
% \end{abstract}