% !TEX root = ../ac_paper.tex

\begin{abstract}
Using a long-standing conjecture from combinatorial group theory as our framework, we explore from multiple angles challenges of finding rare instances that carry disproportionately high rewards. Based on lessons learned in the mathematical context of the Andrews-Curtis conjecture, we propose algorithmic improvements that can be relevant in other domains with ultra sparse reward problems. Although our case study can be formulated as a game, its shortest winning sequences may be $10^6$ or $10^9$ times longer compared to those in a game of chess. In the process of our study, we demonstrate that one of the potential counterexamples due to Akbulut and Kirby, whose status escaped direct mathematical methods for decades, is stably AC-trivial.
\end{abstract}