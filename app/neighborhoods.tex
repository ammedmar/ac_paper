% !TEX root = ../ac_paper.tex

\section{Construction of neighborhoods}\label{s:neighborhoods}

\subsection{Neighborhoods for MS series}

We define the $n$-neighborhood of a balanced presentation $\pi$ as the set of all balanced presentations that can be obtained by applying at most $n$ AC-moves to $\pi$.
We used Algorithm \ref{alg:bfs_neigh}, a variation of the Breadth-First Search (BFS) algorithm, to generate $5$-neighborhoods of presentations in the Miller–Schupp series.
We did this in two settings: in the first, we considered the presentation as a tuple of relators (meaning that the order of relators is important), and in the second, we considered the presentation as a set of relators (meaning that the order of relators is not important; implemented as a tuple of relators in lexicographic order).

\subsection{Neighborhoods of the identity} For any $\ell \in \{3, \dots, 14\}$, we constructed a neighborhood of the identity using the same BFS algorithm. This neighborhood contains all presentations that can be connected to the identity via a path of AC moves, where each presentation in the path has a length less than or equal to $\ell$.

\begin{algorithm}
	\caption{Breadth-First Search Algorithm Bounded by Size}\label{alg:bfs_1}
	\begin{algorithmic}[1]
		\State \textbf{Input:} A balanced presentation $\pi$, maximal size of presentation $n$
		\State \textbf{Output:} Set of enumerated presentations connected to the starting presentation that are achievable without exceeding the size limit, and set of edges with filtrations
		\State Initialize a queue $Q$, set of visited nodes $visited$, and numerical map $name$ that will enumerate presentations
		\State Mark $\pi$ as visited, put it into queue $Q$, and assign it the number $0$
		\While{$Q$ is not empty}
		\State $u \gets $ top of $Q$ \Comment{Remove the front node of $Q$}
		\For{every AC move $m$}
		\State $child \gets m(u)$
		\If{$child$'s size $\leq n$ and $child$ is not visited}
		\State Put $child$ in $Q$ and mark it as visited
		\State Assign $child$ the next available number
		\EndIf
		\If{$child$'s size $\leq n$ and $u$'s number is smaller than $child$'s number}
		\State Return edge $(u, child)$ with proper filtration
		\EndIf
		\EndFor
		\EndWhile
	\end{algorithmic}
\end{algorithm}

\begin{algorithm}
	\caption{Breadth-First Search Algorithm Bounded by Number of Steps}\label{alg:bfs_neigh}
	\begin{algorithmic}[1]
		\State \textbf{Input:} A balanced presentation $\pi$, positive integer $n$
		\State \textbf{Output:} $n$-neighborhood of $\pi$
		\State Initialize a queue $Q$, set of visited nodes $visited$, and numerical map $dist$ that represents the minimal number of AC-moves needed to transform $\pi$ into a given presentation
		\State Mark $\pi$ as visited, put it into queue $Q$, and set its distance to $0$
		\While{$Q$ is not empty}
		\State $u \gets $ top of $Q$ \Comment{Remove the front node of $Q$}
		\For{every AC move $m$}
		\State $child \gets m(u)$
		\If{$dist[u] < n$ and $child$ is not in $visited$}
		\State Put $child$ in $Q$ and mark it as visited
		\State Set $dist[child] := dist[u] + 1$
		\EndIf
		\EndFor
		\EndWhile
		\Return set $visited$
	\end{algorithmic}
\end{algorithm}