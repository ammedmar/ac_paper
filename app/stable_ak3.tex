% !TEX root = ../ac_paper.tex

\subsection{Proof of Theorem \ref{thm:stableAK3}}
%\section{Stable AC-trivialization of $\AK(3)$}
\label{sec:stable_ak3}

\subsubsection*{Moved from S2: The Stable Andrews--Curtis Conjecture}\label{sec:stable_ac}

The Andrews--Curtis conjecture has many known variants.
\fixme{include references} We briefly explored one of these variants, namely the ``stable" or ``weak" Andrews--Curtis conjecture.
The stable version allows two more transformations in addition to the usual AC moves:.

\begin{enumerate}[label=(AC\arabic*)]
	\setcounter{enumi}{3}
	\item Include a new generator and a trivial relator, i.e. replace $\angles{x_1, \dots, x_n \mid r_1, \dots, r_n}$ by $\angles{x_1, \dots, x_n, x_{n+1} \mid r_1, \dots, r_n, x_{n+1}}$.
	\item Remove a trivial relator and the corresponding generator, i.e. the inverse of transformation (4).
\end{enumerate}
If two balanced presentations of the trivial group are related by a sequence of AC transformations (AC1) through (AC5), we say that they are \textit{stably AC-equivalent}.
The stable Andrews--Curtis conjecture states that any balanced presentation is stably AC-equivalent to the trivial presentation.

To the best of our knowledge, the shortest potential counterexample to the standard Andrews--Curtis conjecture, $\AK(3)$, was also believed to be a potential counterexample to the stable Andrews--Curtis conjecture before this work.
We show in \autoref{sec:search} that $\AK(3)$ is in fact stably AC-trivial.
Our proof builds on the following result.

\begin{theorem*}[Myasnikov, Myasnikov, and Shpilrain, \cite{MMS}]\label{theorem:MMS}
	Using the notation $[a, b] = a b a^{-1} b^{-1}$ and $[a, b, c] = [[a, b], c]$, any presentation of the following form is a presentation of the trivial group:
	\[
	\langle x, y, z \mid x = z \cdot [y^{-1}, x^{-1}, z], y = x \cdot [y^{-1}, x^{-1}, z^{-1}] \cdot [z^{-1}, x], w \rangle,
	\]
	where $w$ is a word in $x$, $y$, and $z$ whose exponent sum on $x$, $y$, and $z$ equals $\pm 1$.
\end{theorem*}

They obtained these presentations by applying Reidemeister moves to the knot diagram of the unknot.
Reidemeister moves applied to a knot diagram are known to give stably AC-equivalent Wirtinger presentations of the knot group \cite{WADA1994241}.
Thus the presentations in the theorem are all stably AC-trivial.

For $w = x^{-1}yz$, they reduced the presentation to the following length 25 presentation with two generators by eliminating $z$,
%\footnote{
	%They used a computer program to further reduce this presentation to a length 14 presentation,
	%\[
	%\angles{x, y \mid x y x^{-2} = y x^{-1} y, xy^2 x = y x y}.
	%\]
	%We focus on the length 25 presentation as it follows clearly from the presentations in the theorem.
	%}
\[
\angles{ x, y \mid
	x^{-1}y^{-1}xy^{-1}x^{-1}yxy^{-2}xyx^{-1}y,
	y^{-1}x^{-1}y^2x^{-1}y^{-1}xyxy^{-2}x }.
\]
We discovered a sequence of AC transformations that relates this presentation to $\AK(3)$.
(See \autoref{sec:stable_ak3} for this sequence of moves.)
This also makes $\AK(3)$ the shortest stably AC-trivial presentation that is not yet known to be AC-trivial.


It is plausible that by varying $w$ one can show that other presentations of the Akbulut--Kirby series (or the Miller--Schupp series) are also stably AC-trivial.
We leave this question for future work.

\subsubsection{Moved from S3}

\subsubsection{The Stable Andrews--Curtis Conjecture}

We applied the greedy and breadth-first search algorithms to trivialize the length 25 presentation mentioned in \autoref{sec:stable_ac}:
\[
\angles{ x, y \mid
	x^{-1}y^{-1}xy^{-1}x^{-1}yxy^{-2}xyx^{-1}y,
	y^{-1}x^{-1}y^2x^{-1}y^{-1}xyxy^{-2}x }.
\]
We placed a cutoff of a maximum of 1 million nodes to visit for each of our search algorithms and allowed the length of each relator to increase up to $15$.
Greedy search found a path connecting this presentation to $\AK(3)$, while breadth-first search could only reduce the presentation's length to $14$.
We repeated the search process with breadth-first search with a cutoff of 5 million nodes.
It failed to reduce the presentation length any further.
The sequence of moves discovered by greedy search is given in \autoref{sec:stable_ak3}.

\subsubsection{Old appendix}

Here we provide a sequence of AC transformations that relates the length-25 stably AC-trivial example of \cite{MMS},
\[
\angles{x, y \mid xyx^{-2}y^{-1} xy^{-1}, x^{-1} y^{-1} x y^2 x y^{-1}}.
\]
to $\AK(3)$. We discovered this sequence of transformations using the greedy search algorithm of \autoref{sec:search}. The reader may confirm with the help of a computer program or by hand that this sequence relates the two presentations. To describe the sequence, let us first recall the definitions of AC$'$ moves discussed in \autoref{sec:search},
\begin{enumerate}[label=(AC$'$\arabic*)]
	\item Substitute some $r_i$ by $r_i r_j^{\pm 1}$ for $i \neq j$.
	\item Change some $r_i$ to $g r_i g^{-1}$ where $g$ is a generator or its inverse.
\end{enumerate}

These are 12 transformations, which we list below.
\[
\begin{aligned}
h_1 &:= \  r_2 \rightarrow r_2 r_1 \\
h_2 &:= \ r_1 \rightarrow r_1 r_2^{-1} \\
h_3 &:= \ r_2 \rightarrow r_2 r_1^{-1} \\
h_4 &:= \ r_1 \rightarrow r_1 r_2 \\
h_5 &:= \ r_2 \rightarrow x^{-1} r_2 x \\
h_6 &:= \ r_1 \rightarrow y^{-1} r_1 y \\
h_7 &:= \ r_2 \rightarrow y^{-1} r_2 y \\
h_8 &:= \ r_1 \rightarrow x r_1 x^{-1} \\
h_9 &:= \ r_2 \rightarrow x r_2 x^{-1} \\
h_{10} &:= \ r_1 \rightarrow y r_1 y^{-1} \\
h_{11} &:= \ r_2 \rightarrow y r_2 y^{-1} \\
h_{12} &:= \ r_1 \rightarrow x^{-1} r_1 x
\end{aligned}
\]

The sequence of moves that connects the length-25 presentation to $\AK(3)$ is as follows.
\[
\begin{aligned}
& h_9 \cdot h_7 \cdot h_4 \cdot h_8 \cdot h_{11} \cdot h_5 \cdot h_{11} \cdot h_9 \cdot h_3 \cdot h_{10} \cdot h_{12} \cdot h_7 \cdot h_7 \cdot h_9 \cdot h_{11} \cdot h_5 \cdot h_3 \cdot h_5 \cdot \\
& h_4 \cdot h_3 \cdot h_{12} \cdot h_5 \cdot h_7 \cdot h_7 \cdot h_1 \cdot h_9 \cdot h_{11} \cdot h_8 \cdot h_3 \cdot h_5 \cdot h_{10} \cdot h_2 \cdot h_6 \cdot h_{12} \cdot h_9 \cdot h_7 \cdot \\
& h_5 \cdot h_{11} \cdot h_{10} \cdot h_3 \cdot h_8 \cdot h_{11} \cdot h_9 \cdot h_2 \cdot h_{10} \cdot h_{12} \cdot h_5 \cdot h_7 \cdot h_9 \cdot h_{11} \cdot h_1 \cdot h_9 \cdot h_8
\end{aligned}
\]
This sequence should be read from left to right; first apply $h_9$, then $h_7$, and so forth. This follows the standard convention of many programming languages, which iterate over lists from left to right by default. The length of the presentation did not exceed $25$ during its path to $\AK(3)$. We have not verified if this is the shortest path between the two presentations.

%\[
%\begin{array}{cc}
%[(9, 25), (7, 25), (4, 17), (8, 17), (11, 17), (5, 17), (11, 17), (9, 17), (3, 16), (10, 16), \\
 %(12, 16), (7, 16), (7, 16), (9, 16), (11, 16), (5, 16), (3, 15), (5, 15), (4, 19), (3, 14), (12, 14), \\
 %(5, 14), (7, 16), (7, 18), (1, 19), (9, 19), (11, 19), (8, 19), (3, 18), (5, 18), (10, 18), (2, 15), (6, 15), \\
 %(12, 15), (9, 15), (7, 15), (5, 15), (11, 15), (10, 15), (3, 15), (8, 15), (11, 15), (9, 15), (2, 16),\\
  %(10, 16), (12, 16), (5, 16), (7, 16), (9, 16), (11, 16), (1, 13), (9, 13), (8, 13)]
%\end{array}
%\]

