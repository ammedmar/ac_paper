% !TEX root = ../ac_paper.tex

\subsection{Proof of Theorem \ref{thm:stableAK3}}
%\section{Stable AC-trivialization of $\AK(3)$}
\label{sec:stable_ak3}

%\subsubsection*{Moved from S2: The Stable Andrews--Curtis Conjecture}\label{sec:stable_ac}

As mentioned in \autoref{sec:intro}, one nice byproduct of our analysis is that the shortest mysterious AC presentation, namely $\AK(3)$, is stably AC-trivial. The goal of this part is to present a proof of this statement.

First, in order to make this part of the paper self-contained, let us remind the reader that the term ``stable'' (a.k.a. ``weak'') refers to one of many variants of the Andrews--Curtis conjecture, see e.g. \cite{MMS,Meier2016,Bagherifard2021}, where in addition to the usual AC moves one is allowed to use two more transformations:
\begin{enumerate}[label=(AC\arabic*)]
	\setcounter{enumi}{3}
	\item Include a new generator and a trivial relator, i.e. replace $\angles{x_1, \dots, x_n \mid r_1, \dots, r_n}$ by $\angles{x_1, \dots, x_n, x_{n+1} \mid r_1, \dots, r_n, x_{n+1}}$.
	\item Remove a trivial relator and the corresponding generator, i.e. the inverse of (AC4).
\end{enumerate}

\begin{definition}
If two balanced presentations of the trivial group are related by a sequence of AC transformations (AC1) through (AC5), we say that they are \textit{stably AC-equivalent}.
\end{definition}
The stable Andrews--Curtis conjecture states that any balanced presentation is stably AC-equivalent to the trivial presentation.
To the best of our knowledge, prior to this work, the shortest potential counterexample to the standard Andrews--Curtis conjecture, $\AK(3)$, was also a potential counterexample to the stable Andrews--Curtis conjecture. Our proof that $\AK(3)$ is stably AC-trivial builds on the following result.

\begin{theorem*}[Myasnikov, Myasnikov, and Shpilrain, \cite{MMS}]\label{theorem:MMS}
	Using the notation $[a, b] = a b a^{-1} b^{-1}$ and $[a, b, c] = [[a, b], c]$, any presentation of the following form is a presentation of the trivial group:
	\[
	\langle x, y, z \mid x = z \cdot [y^{-1}, x^{-1}, z], y = x \cdot [y^{-1}, x^{-1}, z^{-1}] \cdot [z^{-1}, x], w \rangle,
	\]
	where $w$ is a word in $x$, $y$, and $z$ whose exponent sum on $x$, $y$, and $z$ equals $\pm 1$. Moreover, all such presentations are stably AC-trivial.
\end{theorem*}

These presentations are obtained by applying Reidemeister moves to the knot diagram of the unknot and using the fact that Reidemeister moves applied to a knot diagram give stably AC-equivalent Wirtinger presentations of the knot group, cf. \cite{WADA1994241}.

For $w = x^{-1}yz$, one of the relators eliminates the generator $z$, resulting in the following length 25 presentation with two generators:
%\footnote{
	%They used a computer program to further reduce this presentation to a length 14 presentation,
	%\[
	%\angles{x, y \mid x y x^{-2} = y x^{-1} y, xy^2 x = y x y}.
	%\]
	%We focus on the length 25 presentation as it follows clearly from the presentations in the theorem.
	%}
\[
\angles{ x, y \mid
	x^{-1}y^{-1}xy^{-1}x^{-1}yxy^{-2}xyx^{-1}y,
	y^{-1}x^{-1}y^2x^{-1}y^{-1}xyxy^{-2}x }.
\]
We discovered a sequence of AC transformations (AC1)--(AC5) that relates this presentation to $\AK(3)$.
This also makes $\AK(3)$ the shortest stably AC-trivial presentation that is not yet known to be AC-trivial.
It is plausible that by varying $w$ one can show that other presentations of the Akbulut--Kirby series (or the Miller--Schupp series) are also stably AC-trivial.
We leave this question for future work.

Specifically, using search algorithms described earlier in this section we placed a cutoff of a maximum of 1 million nodes to visit for each of our search algorithms and allowed the length of each relator to increase up to $15$.
The greedy search found a path connecting this presentation to $\AK(3)$, while breadth-first search could only reduce the presentation's length to $14$.
We repeated the search process with breadth-first search with a cutoff of 5 million nodes.
It failed to reduce the presentation length any further.

The sequence of moves that connects the length-25 presentation to $\AK(3)$ can be conveniently expressed in terms of the following 12 transformations:
% \[
% \begin{aligned}
% h_1 &:= \  r_2 \rightarrow r_2 r_1 \\
% h_2 &:= \ r_1 \rightarrow r_1 r_2^{-1} \\
% h_3 &:= \ r_2 \rightarrow r_2 r_1^{-1} \\
% h_4 &:= \ r_1 \rightarrow r_1 r_2 \\
% h_5 &:= \ r_2 \rightarrow x^{-1} r_2 x \\
% h_6 &:= \ r_1 \rightarrow y^{-1} r_1 y \\
% h_7 &:= \ r_2 \rightarrow y^{-1} r_2 y \\
% h_8 &:= \ r_1 \rightarrow x r_1 x^{-1} \\
% h_9 &:= \ r_2 \rightarrow x r_2 x^{-1} \\
% h_{10} &:= \ r_1 \rightarrow y r_1 y^{-1} \\
% h_{11} &:= \ r_2 \rightarrow y r_2 y^{-1} \\
% h_{12} &:= \ r_1 \rightarrow x^{-1} r_1 x
% \end{aligned}
% \]
\[
\begin{aligned}
h_1 &= \  r_2 \rightarrow r_2 r_1, & \quad h_5 &= \ r_2 \rightarrow x^{-1} r_2 x, & \quad h_9 &= \ r_2 \rightarrow x r_2 x^{-1}, \\
h_2 &= \ r_1 \rightarrow r_1 r_2^{-1}, & \quad h_6 &= \ r_1 \rightarrow y^{-1} r_1 y, & \quad h_{10} &= \ r_1 \rightarrow y r_1 y^{-1}, \\
h_3 &= \ r_2 \rightarrow r_2 r_1^{-1}, & \quad h_7 &= \ r_2 \rightarrow y^{-1} r_2 y, & \quad h_{11} &= \ r_2 \rightarrow y r_2 y^{-1}, \\
h_4 &= \ r_1 \rightarrow r_1 r_2, & \quad h_8 &= \ r_1 \rightarrow x r_1 x^{-1}, & \quad h_{12} &= \ r_1 \rightarrow x^{-1} r_1 x, \\
\end{aligned}
\]
among which a careful reader can recognize moves (AC$'$1) and (AC$'$2) introduced in \autoref{sec:AC}. Expressed in terms of the moves $h_i$, the desired sequence has length 53 and looks as follows:
\[
\begin{aligned}
& h_9 \cdot h_7 \cdot h_4 \cdot h_8 \cdot h_{11} \cdot h_5 \cdot h_{11} \cdot h_9 \cdot h_3 \cdot h_{10} \cdot h_{12} \cdot h_7 \cdot h_7 \cdot h_9 \cdot h_{11} \cdot h_5 \cdot h_3 \cdot h_5 \cdot \\
& h_4 \cdot h_3 \cdot h_{12} \cdot h_5 \cdot h_7 \cdot h_7 \cdot h_1 \cdot h_9 \cdot h_{11} \cdot h_8 \cdot h_3 \cdot h_5 \cdot h_{10} \cdot h_2 \cdot h_6 \cdot h_{12} \cdot h_9 \cdot h_7 \cdot \\
& h_5 \cdot h_{11} \cdot h_{10} \cdot h_3 \cdot h_8 \cdot h_{11} \cdot h_9 \cdot h_2 \cdot h_{10} \cdot h_{12} \cdot h_5 \cdot h_7 \cdot h_9 \cdot h_{11} \cdot h_1 \cdot h_9 \cdot h_8.
\end{aligned}
\]
This sequence should be read from left to right; first apply $h_9$, then $h_7$, and so forth. This follows the standard convention of many programming languages, which iterate over lists from left to right by default. The length of the presentation did not exceed $25$ during its path to $\AK(3)$. We do not know if this is the shortest path between the two presentations.

%\[
%\begin{array}{cc}
%[(9, 25), (7, 25), (4, 17), (8, 17), (11, 17), (5, 17), (11, 17), (9, 17), (3, 16), (10, 16), \\
 %(12, 16), (7, 16), (7, 16), (9, 16), (11, 16), (5, 16), (3, 15), (5, 15), (4, 19), (3, 14), (12, 14), \\
 %(5, 14), (7, 16), (7, 18), (1, 19), (9, 19), (11, 19), (8, 19), (3, 18), (5, 18), (10, 18), (2, 15), (6, 15), \\
 %(12, 15), (9, 15), (7, 15), (5, 15), (11, 15), (10, 15), (3, 15), (8, 15), (11, 15), (9, 15), (2, 16),\\
  %(10, 16), (12, 16), (5, 16), (7, 16), (9, 16), (11, 16), (1, 13), (9, 13), (8, 13)]
%\end{array}
%\]

