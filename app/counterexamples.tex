% !TEX root = ../ac_paper.tex

\section{Potential Counterexamples}

While we focused on a subset of presentations from the Miller-Schupp series in this paper, there are a 

We mentioned some potential counterexamples of the Andrews-Curtis conjecture in the main text in section 2.2. Here, we mention some other potential counterexamples. 

A third series of potential counterexamples is due to Gordon \cite{Brown}:
\[
\G(m, n, p, q) \colon \angles{x, y \mid x = [x^m, y^n],\ y = [y^p, x^q]}.
\]
where $m,n,p,q \in \mathbb{Z}$.
All examples of this series with the total relator length of 14 are known
to be AC-trivializable \cite{Bowman-McCaul}
\footnote{Perform the same exercise as in the footnote above for this series.}
\newline

There is also a length-14 presentation in \cite{MMS} for which the authors could not reduce the length of the relators using AC moves.
\[
\angles{x, y \mid xyx^{-2}y^{-1} xy^{-1}, x^{-1} y^{-1} x y^2 x y^{-1}}.
\]


There also exist other generalizations of the conjecture for other kinds of groups; some of which have been proved.
In particular, the conjecture holds true for finite groups and soluble groups \cite{Borovik, Guyot}.
I think these versions will not be important for us.
\newline

Finally, there exists a notion of "elementary M-transformations" which could be useful \cite{BurnsI, BurnsII}.
Each elementary M-transformation is a composition of AC moves, such that trivializing a presentation requires dramatically less M-transformations compared to the number of AC moves.
For example, trivializing $\AK(2)$ requires 14 AC moves, but only 2 elementary M-transformations.
\fixme{Explain M-transformations a bit here.}
