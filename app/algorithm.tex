% !TEX root = ../ac_paper.tex

\section{Algorithm for Generation Language Modeling Dataset\label{app:algorithm}}

%In this appendix we discuss the algorithm, \autoref{alg:apply_ac_moves}, used to generate the training dataset for our Transformer model.

Algorithm  generates presentations of increasing lengths in $n$ phases.
In the $i$-th phase, presentations with maximum relator length $l_i$ are generated, where $l_i$ is sampled from a discrete uniform distribution with bounds $l + i l_{\text{inc}} $ and $l + (i+1) l_{\text{inc}}$.
Here, $l$ is the maximum of the lengths of the two relators in the initial Miller-Schupp presentation $R_0$ (and thus the minimum value of maximum relator length in the algorithm), and $l_{\text{inc}}$ is defined in terms of the maximum possible relator length $l_{\text{max}}$ we wish to obtain in the last phase as $l_{\text{inc}} = (l_{\text{max}}-l)/n$.
In the $i$-th phase, $T$ AC moves are applied to a presentation generated in the $(i-1)$-th phase.
In our work, we chose $l_{\text{max}}=128$, $n=128$, $m=12$, and $T=1000$.
\footnote{Specify much earlier in the paper, and here as well, that whenever we apply AC moves, we set a maximum length up to which each relator is allowed to grow.
Any AC move that would result in a presentation of longer relator length acts trivially.
Maybe instead of writing $A \sim \text{AC Moves}$, there should be a notation like $(AC)_l$ when $l$ is specified to be the maximum relator length.}


The main reason for choosing this algorithm is so that our GPT model sees roughly the same amount of presentations of each length and is, hence, not biased towards shorter or longer presentations.


\begin{algorithm}
	\caption{Generate AC Cousins}\label{alg:apply_ac_moves}
	\begin{algorithmic}
		\Require $R_0$, $l_{\text{max}}$, $n$, $m$, $T$
		\State $L \gets \{ \}$ \Comment{Initiate the list $L$ of all presentations}

		\State $l \gets \max(l_1, l_2)$

		\Comment{Set $l_1$, $l_2$ are the lengths of relators of $R_0$}

		\State $l_{\text{inc}} \gets (l_{\text{max}}-l)/n$

		\Comment{The increment by which the maximum relator length increases in each phase}

		\For {$i = 1, 2, \dots, n$}

		\Comment{Loop over $n$ phases}
		\For {$j = 1, 2, \dots, m$}

		\Comment{Generate $m$ presentations in each phase}
		\State $l_i \sim \mathcal{U}(l + i l_{\text{inc}} , l + (i+1) l_{\text{inc}})$

		\Comment{Set the maximum relator length for $i$-th phase}
		\If{i is 0}
		\State $R \gets R_0$
		\Else
		\State $R \gets L[(i-1) \times m+j]$
		\EndIf

		\Comment{Set $R$ to a presentation of the previous phase if $i > 0$, else set it to the initial presentation $R_0$}


		\For{$k = 1, 2, \dots, T$}

		\Comment{Apply $T$ randomly selected AC-moves}
		\State A $\sim$ AC Moves
		\State $R \gets A \cdot R$

		\EndFor

		\State $L \gets R$

		\Comment{Add the resultant presentation $R$ to $L$}

		\EndFor
		\EndFor
	\end{algorithmic}
\end{algorithm}
