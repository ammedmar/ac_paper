% !TEX root = ../ac_paper.tex

\section{Language Modeling Dataset Generation \label{app:algorithm}}

This appendix describes the method, \autoref{alg:apply_ac_moves}, used to generate the training and evaluation datasets for the Transformer model, as referenced in \autoref{sec:lm}. Our aim was to create datasets featuring presentations of varying lengths. We began with a presentation \(P_0\) from the Miller--Schupp series, where \(n, \length(w) \leq 7\), setting a maximum relator length \(l_{\text{max}} = 128\). Presentations were generated in \(n=128\) phases, each phase allowing a maximum relator length \(l_i \sim \mathcal{U}(l + i \cdot l_{\text{inc}}, l + (i+1) \cdot l_{\text{inc}})\). Here, \(l\) represents the longest relator length in \(P_0\) and \(l_{\text{inc}} = (l_{\text{max}} - l)/n\) is the incremental increase per phase. In each phase, we selected a presentation \(P\) from the previous phase and applied \(N=1000\) AC$'$ moves. Any AC$'$ move that exceeded the length \(l_i\) resulted in no change.


We repeated this for all 1190 presentations in the Miller--Schupp series, ultimately producing approximately 1.8 million balanced presentations. The length distribution of these presentations is detailed in \autoref{fig:gpt_data}.

\begin{algorithm}
    \caption{Transformer Dataset Generation}
    \label{alg:apply_ac_moves}
    \begin{algorithmic}[1]
        \State \textbf{Input:}
     	\begin{itemize}
	\item[]     $P_0$  --- an initial presentation with $l$ as the length of the longest relator
          \item[]   $n$  --- number of phases
           \item[]  $m$  --- number of presentations in each phase
           \item[]  $N$  --- number of AC$'$ moves to apply in each phase
         \item[]    $l_{\text{max}}$  --- upper bound on presentation lengths in the dataset
          \end{itemize}
         \State \textbf{Output:}
         \begin{itemize}
         \item[]         Dataset (the final collection of presentations)
         \end{itemize}
        \State Dataset $\gets \emptyset$ \Comment{Initialize the dataset of all presentations}
        \State $l_{\text{inc}} \gets (l_{\text{max}} - l) / n$ \Comment{Increment for the maximum relator length per phase}
        \For{$i = 0$ \textbf{to} $n-1$} \Comment{Loop over each phase}
            \For{$j = 1$ \textbf{to} $m$} \Comment{Generate $m$ presentations for each phase}
                \State $l_i \sim \mathcal{U}(l + i \cdot l_{\text{inc}}, l + (i+1) \cdot l_{\text{inc}})$
                \Comment{Sample maximum relator length}
		\State $P \gets (i = 1) \ ? \ P_0 \ : \ \text{Dataset}[(i-1) \cdot m + j - 1]$
                \For{$k = 1$ \textbf{to} $N$} \Comment{Apply $N$ AC$'$ moves with relator length $l_i$}
                    \State $A \sim \text{AC}' \text{ Moves}$
                    \State $P \gets A \cdot P$
                \EndFor
                \State Dataset $\gets$ Dataset $\cup \{P\}$
                \Comment{Add the presentation $P$ to the Dataset}
            \EndFor
        \EndFor
    \end{algorithmic}
\end{algorithm}



